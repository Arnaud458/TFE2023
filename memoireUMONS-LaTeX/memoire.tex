\documentclass[12pt,a4paper,oneside, titlepage]{report}

\usepackage{times}
\usepackage[frenchb]{babel}
\usepackage{hyperref} 
\usepackage[utf8]{inputenc}
%\usepackage[T1]{fontenc}
%\usepackage{amsmath}
%\usepackage{amsfonts}
%\usepackage{amscd}
%\usepackage{amstext}
%\usepackage{amssymb}
%\usepackage{bar}
\usepackage{color}
%\usepackage{mathrsfs}
\usepackage{graphicx}
%\usepackage{calligra}
%\usepackage{amsthm}
%\usepackage{multirow}
%\usepackage{tabularx}
%\usepackage{layout}
%\pagestyle{headings}
\usepackage{fancyhdr}
\pagestyle{fancy}

%\setlength{\textheight}{630pt}
%\setlength{\footskip}{30pt}
\newtheorem{defi}{D\'efinition}[section]
\newtheorem{note}{Note}[section]
\newtheorem{proprietet}{Propri\'et\'e}[section]
\newtheorem{exemple}{Exemple}[section]
\newtheorem{corollaire}{Corollaire}[section]
\newtheorem{rem}{Remarque}[section]
\newtheorem{thm}{Th\'eor\`eme}[section]
\newtheorem{illustration}{Illustration}[section]
\newenvironment{demonstration}{\begin{proof}[\textnormal{\textbf{Preuve.}}]}{\end{proof}}
\definecolor{gris}{gray}{0.45}
\setlength{\parindent}{1cm}
\newcommand{\textcalli}[1]{{\small{\textbf{$\negmedspace$\calligra #1}}}}

\renewcommand{\chaptermark}[1]{\markright{\thechapter\ #1}}
%\renewcommand{\sectionmark}[1]{\markright{\thesection\ #1}}
\fancyhf{} % supprime les en-têtes et pieds prédéfinis
\fancyhead[R]{\thepage}% Left Even, Right Odd
\fancyhead[L]{\textsl{\leftmark}} % Left Odd
%\fancyhead[RE]{\textsl{\leftmark}} % Right Even
\renewcommand{\headrulewidth}{0pt}% filet en haut de page
\renewcommand{\footrulewidth}{0pt} % pas de filet en bas
\fancypagestyle{plain}{ % pages de tetes de chapitre
\fancyhead{} % supprime l’entete
\fancyhead[R]{\thepage}
\renewcommand{\headrulewidth}{0pt} % et le filet
}

\begin{document}

%newpage
%\thispagestyle{empty}
%\null
%\newpage
\pagenumbering{roman}
\chapter*{Remerciements}
\renewcommand{\leftmark}{REMERCIEMENTS}
%\addcontentsline{toc}{chapter}{Remerciements}

Nous remercions ...\\

\newpage
\renewcommand{\leftmark}{TABLE DES MATI\`{E}RES}
\thispagestyle{fancy}
\tableofcontents


\newpage
\pagenumbering{arabic}
\renewcommand{\leftmark}{INTRODUCTION}
\chapter{Introduction}

LoRa signiffie Long range. Protocal faible poussance et faible débit. Dans l'IOT besoin de matérielle peu gourmant en énergie, conséquence des appareils a aible capacité CPU, à aible capacité de mémoire.

Besoin de faire de la communication long range ou short range.

Lora basé sur CSS, chirp spread spectrum.

Dans premier chapitre analyse de la couche physique de Lora. Décompostion de chaque étape: coding, ineterleaving, whitening et modulation CSS. 

dans le second chapitre ... 


\newpage


%ICI COMMENCE LE CHAPITRE 2
\chapter{SDR implementation of LoRa}\label{ch:1}
\renewcommand{\leftmark}{CHAPITRE \thechapter.~~Physical LoRa}

\section{channel coding}

Le but du codage est de rendre la transmission robuste au bruit via une chaine de communication, utilisant de la redondance.

Redondance générée pour un paramètre de débit de codage.


\subsection{coding}

formule mathématique.

\subsection{deconding}

Deux méthode :

\begin{itemize}
\item hard deconding
\item sot decoding
\end{itemize}

\section{channel whitening}

transformer la séquence pour rendre les bits de la séquence faiblement corrélé.

\subsection{whitening}

ajout d'une séquence pseudo random à l'output de l'encodeur.

\subsection{dewithening}

codeword associé à tout les bloc 0 est le codeword des bloc 0, donc b = 0. La séquence whiteneed bw = w.

Deux méthode :

\begin{itemize}
\item hard dewhitening. Considérer whitening et dewhitening similaire, différence c'est que le dernier prend une séquence blanchie et sort la séquence originale.
\item sot dewhitening. 
\end{itemize}


\section{channel interleaving}

\subsection{interleaving}

\subsection{deinterleaving}

\section{CSS modulation}

\subsection{modulation}

\subsection{demodulation}

%ICI COMMENCE LE CHAPITRE 3
\chapter{LoRaWAN}
\renewcommand{\leftmark}{CHAPITRE \thechapter.~~Titre du troisième chapitre}

%ICI COMMENCE LE DERNIER CHAPITRE
\chapter*{Conclusion}
\addcontentsline{toc}{chapter}{Conclusion}
\renewcommand{\leftmark}{CONCLUSION}

Mettez votre conclusion ici.  Dressez le bilan de votre travail effectué, en prenant du recul. Discuter de si vous avez bien réussi les objectifs du travail ou non. Présentez les perspectives futurs.


%Le style bibliographique utilisŽ
\bibliographystyle{latex8}

%Le fichier .bib uitilisŽ
\bibliography{biblio}

\newpage
\appendix
\addcontentsline{toc}{chapter}{Annexes}

\chapter{Premi\`ere annexe}
\renewcommand{\leftmark}{ANNEXE \thechapter.~~Premi\`ere annexe}
\label{annexe1}

\chapter{Deuxi\`eme annexe}
\renewcommand{\leftmark}{ANNEXE \thechapter.~~Deuxi\`eme annexe}
\label{annexe2}

%%%%%%%FIN-ANNEXES%%%%%%%%%%
\end{document}
