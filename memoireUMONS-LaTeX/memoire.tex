\documentclass[12pt,a4paper,oneside, titlepage]{report}

\usepackage{times}
\usepackage[frenchb]{babel}
\usepackage[breaklinks]{hyperref} 
\usepackage[utf8]{inputenc}
%\usepackage[T1]{fontenc}
\usepackage{amsmath}
\usepackage{listings}
\usepackage{adjustbox}
\usepackage{changepage}
\usepackage{acronym}
%\usepackage{amsfonts}
%\usepackage{amscd}
%\usepackage{amstext}
%\usepackage{amssymb}
%\usepackage{bar}
\usepackage{color}
\usepackage{xcolor}
%\usepackage{mathrsfs}
\usepackage{graphicx}
%\usepackage{calligra}
%\usepackage{amsthm}
%\usepackage{multirow}
%\usepackage{tabularx}
%\usepackage{layout}
%\pagestyle{headings}
\usepackage{fancyhdr}
\usepackage{subcaption}
\pagestyle{fancy}

%\setlength{\textheight}{630pt}
%\setlength{\footskip}{30pt}
\newtheorem{defi}{D\'efinition}[section]
\newtheorem{note}{Note}[section]
\newtheorem{proprietet}{Propri\'et\'e}[section]
\newtheorem{exemple}{Exemple}[section]
\newtheorem{corollaire}{Corollaire}[section]
\newtheorem{rem}{Remarque}[section]
\newtheorem{thm}{Th\'eor\`eme}[section]
\newtheorem{illustration}{Illustration}[section]
\newenvironment{demonstration}{\begin{proof}[\textnormal{\textbf{Preuve.}}]}{\end{proof}}
\definecolor{gris}{gray}{0.45}
\setlength{\parindent}{1cm}

\lstdefinestyle{pythonstyle}{
    language=Python,
    basicstyle=\small\ttfamily,
    keywordstyle=\color{blue},
    commentstyle=\color{green!60!black},
    stringstyle=\color{orange},
    numbers=left,
    numberstyle=\tiny,
    numbersep=5pt,
    breaklines=true,
    frame=tb,
    columns=fullflexible,
    backgroundcolor=\color{gray!10},
    linewidth=\linewidth,
    xleftmargin=0pt,
    xrightmargin=0pt,
    showstringspaces=false,
    captionpos=b,
}

\lstdefinestyle{cppstyle}{
    language=C++,
    basicstyle=\small\ttfamily,
    keywordstyle=\color{blue},
    commentstyle=\color{green!60!black},
    stringstyle=\color{orange},
    numbers=left,
    numberstyle=\tiny,
    numbersep=5pt,
    breaklines=true,
    frame=tb,
    columns=fullflexible,
    backgroundcolor=\color{gray!10},
    linewidth=\linewidth,
    xleftmargin=0pt,
    xrightmargin=0pt,
    showstringspaces=false,
    captionpos=b,
}

\newcommand{\textcalli}[1]{{\small{\textbf{$\negmedspace$\calligra #1}}}}

\renewcommand{\chaptermark}[1]{\markright{\thechapter\ #1}}
%\renewcommand{\sectionmark}[1]{\markright{\thesection\ #1}}
\fancyhf{} % supprime les en-têtes et pieds prédéfinis
\fancyhead[R]{\thepage}% Left Even, Right Odd
\fancyhead[L]{\textsl{\leftmark}} % Left Odd
%\fancyhead[RE]{\textsl{\leftmark}} % Right Even
\renewcommand{\headrulewidth}{0pt}% filet en haut de page
\renewcommand{\footrulewidth}{0pt} % pas de filet en bas
\fancypagestyle{plain}{ % pages de tetes de chapitre
\fancyhead{} % supprime l’entete
\fancyhead[R]{\thepage}
\renewcommand{\headrulewidth}{0pt} % et le filet
}
\setcounter{secnumdepth}{3}
\begin{document}

\begin{titlepage}

\begin{center}
\textnormal{\Large{Universit\'e de Mons}}\\[0.3em]

\begin{center}
\includegraphics[height=2cm]{UMONS-logo.jpg}
\end{center}

\textnormal{\Large{Facult\'e des Sciences}}\\[0.3em]

\begin{center}
\includegraphics[height=1.7cm]{FS-logo.jpg}
\end{center}


\end{center}
\vspace*{2cm}
\begin{center}
\fbox{
\begin{minipage}{14cm}
\center
\vspace*{0.5cm}\textbf{\LARGE{Identifier des noeuds IoT en espionnant \\ leur signal radio}}\\[0.5em]
\vspace*{0.3cm}
\end{minipage}
}
\end{center}
\vspace*{1.5cm}


 
\large{
\begin{center}
M\'emoire r\'ealis\'e par\\
 Arnaud \textsc{Tulippe Hecq}\\
 \vspace*{1.5cm}
\textnormal{\Large{D\'epartement d'Informatique}}\\[0.3em]
\textnormal{\Large{SERVICE RESEAU ET TELECOMMUNICATION}}
Directeur : Professeur Bruno \textsc{Quoitin}  

\end{center}}

\vfill

\begin{center}
Ann\'ee acad\'emique 2023-2024
\end{center}

\end{titlepage}

%newpage
%\thispagestyle{empty}
%\null
%\newpage
\pagenumbering{roman}
\chapter*{Remerciements}
\renewcommand{\leftmark}{REMERCIEMENTS}
%\addcontentsline{toc}{chapter}{Remerciements}

Je tiens à exprimer toute ma reconnaissance à mon directeur de mémoire, Mr Bruno Quoitin, pour sa disponibilité, ses conseils et ses commentaires ainsi que son soutien et ses encouragements durant cette année académique, mais également durant tout mon parcours universitaire.

\vspace{0.1cm}

Je remercie également Mme Véronique Moeyaert pour son aide sur la compré\-hension de certains concepts de télécommunication.

\vspace{0.1cm}

Enfin, je remercie tous ceux qui, de près ou de loin, ont rendu possible la rédaction de ce travail par leur soutien, leurs conseils ou leurs connaissances qu'ils m'ont transmises.

\newpage
\renewcommand{\leftmark}{TABLE DES MATI\`{E}RES}
\thispagestyle{fancy}
\tableofcontents


\section*{Liste des acronymes}
\renewcommand{\leftmark}{LISTE DES ACRONYMES}
\addcontentsline{toc}{section}{Liste des acronymes}
\begin{acronym}[Bash]
\acro{IoT}{Internet of Things}
\acro{LoRa}{Long Range}
\acro{WAN}{Wide Area Network}
\acro{RFID}{Radio Frequency Identification}
\acro{EPC}{Electronic Product Code}
\acro{PKI}{Public Key Infrastructure}
\acro{CA}{Certificate Authority}
\acro{DDoS}{Distributed Denial of Service}
\acro{RGPD}{Règlement Général sur la Protection des Données}
\acro{RSA}{Rivest-Shawir-Adleman}
\acro{ECC}{Elliptic Curve Cryptography}
\acro{SNR}{Signal to Noise Ratio}
\acro{I/Q}{In phase / Quadrature}
\acro{AM}{Amplitude Modulation}
\acro{FM}{Frequency Modulation}
\acro{PM}{Phase Modulation}
\acro{CFT}{COntinuous Fourier Transform}
\acro{DFT}{Discrete Fuurier Transform}
\acro{FFT}{Fast Fourier Transform}
\acro{LPWAN}{Low Power Wide Area Network}
\acro{ISM}{Industrial Scientific and Medical}
\acro{CSS}{Chirp Spread Spectrum}
\acro{TCP}{Transport layer Protocol}
\acro{HTTP}{Hypertext Transfer Protocol}
\acro{MQTT}{Message Queuing Telmetry Transport}
\acro{MAC}{Media Acces Control}
\acro{OSI}{Open System Interconnection}
\acro{FEC}{Forward Error Correction}
\acro{FSCM}{Frequency Shift Chrip Modulation}
\acro{FSK}{Frequency Shift Keying}
\acro{SF}{Spreading Factor}
\acro{CRC}{Cyclic Redundancy Check}
\acro{RSSI}{Received Signal Strengh Indication}
\acro{TDOA}{Time Difference On Arrival}
\acro{LBT}{Listen Before Talk}
\acro{OTAA}{Over The Air Activation}
\acro{MIC}{Message Integrity Code}
\acro{ECB}{Electronic Codebook}
\acro{ABP}{Activation By Personalisation}
\acro{SDR}{Software Defined Radio}
\acro{DVB-T}{Digital Video Broadcasting Terrestrial}
\acro{RF}{Radio Frequency}
\acro{USB}{Universal Serial Bus}
\acro{LNA}{Low Noise Amplifier}
\acro{LO}{Local Oscillator}
\acro{PLL}{Phase-Locked Loop}
\acro{VGA}{Variable Gain Amplifier}
\acro{ADC}{Analog to Digital Converter}
\acro{DDC}{Digital Down Converter}
\acro{LPF}{Low Pass Filter}
\acro{GFSK}{Gaussian Frequency Shift Keying}
\acro{TX/RX}{Transmitter / Receiver}
\acro{UART}{Universal Asynchronous Receiver Transmitter}
\acro{TOA}{Time On Air}
\acro{IDE}{Integrated Development Environment}
\acro{ACG}{Automcatic Gain Control}
\acro{URH}{Universal Radio Hacker}
\acro{GUI}{Graphical User Interface}
\acro{CR}{Coding Rate}
\acro{BW}{Bandwidth}
\acro{RMS}{Root Means Square}
\acro{DCTF}{Differential Constellation Trace Figure}



\end{acronym}

\pagenumbering{arabic}



\chapter*{Introduction}

\addcontentsline{toc}{chapter}{Introduction}
\renewcommand{\leftmark}{INTRODUCTION}

L'avènement de \ac{IoT} a lancé une nouvelle ère d'appareils connectés, ouvrant de nouvelles possibilités de partage de l'information, d'automatisation et de protection. Bien que le concept lui même soit prometteur, la technologie qui l'accompagne est essentielle.
Les premières technologies utilisées pour l'\ac{IoT} étaient les technologies sans fil déja présentes comme le Wifi ou le Bluetooth. Elles ont cependant plusieurs limitations : une consommation en énergie élevée, une portée restreinte et parfois même un coût d'infrastructure trop important.
Dans ces circonstances, est apparue \ac{LoRa}, une technologie développée en particulier pour l'\ac{IoT}. Sa capacité à gérer les communications longue portée, même dans des environnements urbains très denses, est un grand atout pour le domaine.

\vspace{0.1cm}

L'expansion de L'\ac{IoT} soulève une nouvelle problèmatique de sécurité. Entre autres, l'identification des noeuds au sein des réseaux est essentielle. Il a été dé\-couvert que des noeuds fabriqués avec les mêmes microprocesseurs et modèles d'émetteurs-récepteurs radio peuvent présenter de subtiles particularités dans les caractéristiques de leurs signaux. Cette variabilité intrinsèque de la transmission des signaux radio peut être exploitée pour distinguer les noeuds d’un réseau. En écoutant leurs signaux radio et en analysant leurs signatures distinctes, il devient possible de les identifier.

\vspace{0.1cm}

Ce travail est structuré en trois parties. Le premier          \hyperref[chap1]{chapitre} sert d'aperçu global du signal radio afin d'y développer et rappeler les concepts de base. Ce chapitre présente également les technologies \ac{LoRa} et LoRaWAN à travers leurs caractéristiques et leur pertinence dans l'\ac{IoT}.
Le deuxième chapitre est dédié à l'étude expérimentale du sujet. Les aspects pratiques y seront appliqués, notamment l'utilisation de radio logicielle afin de capturer des signaux radio. Ces signaux seront ensuite analysés et automatisés pour la suite du travail.
Le troisième chapitre présentera la méthode utilisée pour réaliser l'objectif du mémoire, ainsi que son application pratique sur les appareils décrits au chapitre précédent. Enfin le travail sera achevé en concluant sur de potentielles implications plus larges à ce sujet ainsi que des recherches plus approfondies.

\newpage

Afin de bien comprendre les enjeux du mémoire, voici une petit historique de l'évolution des préoccupations dans l'\ac{IoT} au cours des deux décennies précédentes. Les technologies mais également le concept même d'\textit{Internet of Things} ont évolué depuis.

\vspace{0.1cm}

Bien que le concept d'appareils connectés remonte aux années 70, l'avène\-ment de l'Internet of Things arrive en fin de millénaire. Ce concept est associé à la technologie \ac{RFID} \cite{RFID}, qui permet d'utiliser les ondes radios afin d'identifier des objets ou des personnes. Le but initial était de rendre tout objet dans le monde identifiable par un code \ac{EPC} \cite{EPC}, un peu comme un code barre. Durant ces années, plusieurs entreprises lancent leurs premiers appareils connectés. Tout cet enthousiasme pour la connectivité met au second plan les questions de sécurité. Ainsi la première partie du développement de l'\ac{IoT} se concentre surtout sur la qualité de communication entre les objets plutôt que sur leur sécurité.

\vspace{0.1cm}

Vers la fin des années 2000, l'augmentation du nombre d'appareils est si grande qu'elle a atteint tous les domaines de la société. Certains domaines étant plus critiques que d'autres d'un point de vue sécurité (l'énergie, les transports, la santé, etc.), l'intégrité des données, la confidentialité et les accès réseaux deviennent le centre de l'attention. 
Le concept de certificats x.509, initalement dévelop\-pé pour le World Wide Web avant les années 2000, a un regain d'attention dans cette période. Plus largement, la structure de la technologie \ac{PKI}, qui utilise les certificats x.509 \cite{PKI} a été adaptée pour s'intégrer aux problématiques de l'embarqué. Un certificat est un document digital permettant de vérifier l'identité d'une entité, comme d'un appareil, un utilisateur ou une organisation. Il se base sur la liaison d'une clé publique à l'entité établie par une \ac{CA}. La CA agit en tant que tiers de confiance et assure la légitimité de l'information grâce au certificat. Ainsi, les trois axes principaux de la sécurité dans l'\ac{IoT} émergent : l'authentification, l'intégrité des données et la confidentialité.

\vspace{0.1cm}


Vers les années 2010, le nombre d'appareils connectés dépasse le nombre d'êtres humains, forçant une transition vers l'IPv6 tant le nombre d'appareils est élevé et continue d'augmenter. L'information a pris de la valeur et de l'ampleur. Ainsi, viennent se greffer de nouveau enjeux économiques aux enjeux sécuritaires. La quantité de données générées nécessite de revoir le stockage de l'information. C'est ainsi que va apparaitre le edge computing \cite{edge}, qui est une réponse directe aux besoins des architectures de gérer autant de données en périphérie de réseau. Le concept du edge computing vise à effectuer des calculs et des analyses des données directement sur les appareils connectés, plutôt que de les envoyer vers un centre de données centralisé. Cela réduit la latence, améliore l'efficacité du réseau et permet des analyses en temps réel. Le premier malware spécialement centré sur l'\ac{IoT} fait son apparition. Mirai\cite{Mirai} exploite une faille lui permet de récupérer les mots de passe d'appareils afin de s'en servir pour lancer des attaques \ac{DDoS} à grande échelle. En quelques années, l'\ac{IoT} est passé d'un gadget d'entreprise à un véritable enjeu économique et sécuritaire, centré autour de l'information. Les seules perspectives de législation concernant la sécurité de l'Internet of Thing n'apparaitront de tardivement à la fin des années 2010 avec la loi européenne sur le \ac{RGPD}\footnote{loi RGPD: \href{https://commission.europa.eu/law/law-topic/data-protection/data-protection-eu_en}{{https://commission.europa.eu/law/law-topic/data-protection/data-protection-eu-en}}}. Cette loi ne couvre pas la sécurité des appareils mais plutôt l'utilisation des données sur internet en général.

\vspace{0.1cm}

A partir de la fin des années 2010, le stockage et la transmission de données, l'authentification d'appareils ou encore la confidentialité sont au centre des préoc\-cupations. Initialement utilisée dans les cryptomonnaies, la technologie Blockchain est un mécanisme de base de données qui permet un partage transparent des informations au sein d'un réseau. Une base de données Blockchain stocke les données dans des blocs qui sont reliés entre eux dans une chaîne. Les données sont chronologiquement cohérentes, car il n'est pas possible de supprimer ou modifier la chaîne sans le consensus du réseau. Par conséquent, la technologie Blockchain peur servir de livre inaltérable ou immuable pour le suivi des ordres, des paiements, des comptes et d'autres transactions. Le système dispose de mécanismes intégrés qui empêchent les entrées de transactions non autorisées et créent une cohérence dans la vue partagée de ces transactions. L'implémentation de la blockchain pour l'\ac{IoT} confère les avantages suivants\cite{block} : 
\begin{itemize}
\item l'immuabilité. La Blockchain permet de créer un enregistrement immuable de toutes les interactions et communications des appareils. Cet enregistrement peut être utilisé pour détecter et empêcher l'accès non autorisé ou la modification des appareils ou des données dans l'\ac{IoT}.
\item La décentralisation. Il est possible de créer un système décentralisé pour l’authentification et la communication des appareils. Chaque appareil \ac{IoT} se connecte au réseau Blockchain et se voit attribuer une identité numérique unique, qui est vérifiée grâce à l'utilisation de signatures numériques ou de contrats intelligents. Cela élimine le besoin d’une autorité centrale pour authentifier les appareils et annule ainsi les risques de "single point of failure".
\item La confidentialité. La technologie Blockchain peut sécuriser la communication entre les appareils IoT grâce à l'utilisation de la cryptographie à clé publique ou asymétrique. Cela permet l’échange sécurisé d’informations entre appareils sans avoir recours à des intermédiaires.
\end{itemize}

\vspace{0.1cm}

Les menaces de sécurité sont de plus en plus sophistiquées au début des années 2020. Comment faire encore confiance aux infrastructures qui doivent gérer autant d'appareils ? La réponse est de ne plus leur faire confiance. Le zero trust model est donc un modèle basé sur l'absence totale de confiance et une vérification constante, que la demande d'accès provienne de l'intérieur ou de l'extérieur du réseau. Dans un modèle de sécurité classique, une fois qu'un utilisateur ou un appareil accède au réseau interne, on lui fait souvent implicitement confiance, ce qui lui permet une grande liberté d'action au sein du réseau. Le zero trust model suppose cependant que des menaces peuvent exister à la fois à l’intérieur et à l’extérieur du périmètre du réseau et nécessite donc une vérification continue de la confiance. Un modèle qui s'applique sur ce principe devrait contenir les éléments suivants \cite{zero1} :
\begin{itemize}
\item La vérification d'identité. Les utilisateurs et les appareils doivent subir une authentification avant d'accéder à n'importe quel service ou ressource du réseau.
\item Le least privilege access. les permissions sont accordées de manière limitée selon le besoin de l'utilisateur ou de l'appareil.
\item La micro segmentation. Diviser le réseau en segments pour limiter son accès par les appareils.
\item La surveillance en continu. L'analyse du trafic, du comportement et des activités des appareils.
\item Le chiffrage des données.
\end{itemize}

\vspace{0.1cm}


Avec l'arrivée des ordinateurs quantiques dans les prochaines années, les mécanismes de chiffrement basés sur la complexité mathématique comme \ac{RSA} ou \ac{ECC} sont voués à disparaitre  \cite{quantumcrypto}. La puissance de calcul des ordinateurs quantiques est déja considérée comme une véritable menace pour la sécurité informatique. Fort heureusement, c'est également un nouveau champ de possibilités qui s'ouvre pour la sécurité, avec le développement du post quantum cryptography. Un premier protocole résistant aux menaces quantiques, Quantum Key Distribution permet d'établir des canaux de communication entre différents appareils dans l'\ac{IoT}. Ce protocole n'est pas encore en service dans l'\ac{IoT}, mais les premiers tests réalisés en laboratoire sont très prometteurs \cite{qinternet}.

\renewcommand{\leftmark}{NOMINATION DES TECHNOLOGIES}


\chapter{Rappels et nomination des technologies}\label{chap1}

\section{Signal radio}

Un signal est une variation dans l'espace ou dans le temps d'une quantité physique contenant de l'information. Un signal peut être continu ou discret, on le nomme alors respectivement analogique ou numérique. Le type de signal dépend notamment de l'information qu'il contient. Un signal analogique est continue en amplitude, ce qui veut dire qu'il peut contenir un nombre infini de valeur, ainsi que prendre toutes les valeurs possibles, là où un signal numérique contient généralement un nombre fini de valeur (par exemple des 0 et 1).
Les deux catégories ne sont pas incompatible car il est souvent nécessaire en télécommication de pouvoir passer de l'un à l'autre.

\vspace{0.1cm}

L'utilisation de signaux radio en télécomunication confère de nombreux avantages, comme la portée, la vitesse de transmission  ou encore le coût de propagation. Pouvoir transporter de l'information sans avoir recours à du support matériel complet (pas besoin de cable, le signal passe dans l'air) réduit\\ considérablement le cout de la transmission. Ajouté à cela, il est possible d'adapter un signal pour le rendre compatible avec diverse canaux de transmission et de réception, grâce à la modulation. La modulation est une technique permettant de modifier les propriétés du signal lui permettant de transporter de l'information.

\newpage

En télécommunication, les signaux sont des ondes électromagnétiques appelées signal radio. Les signaux comportent de nombreuses caractéristiques qui les déterminent: 

\vspace{0.1cm}



la fréquence, mesurée en Hertz ($Hz$). Elle détermine le nombre de cycle qu'accomplit le signal par seconde. Une onde radio possède une fréquence entre 9kHz et 300GHz.

\vspace{0.1cm}

La largeur de spectre, elle dépent de la fréquence car c'est l'écart entre la plus haute et la plus basse fréquence du signal. Une plus grande largeur permet de transmettre plus d'informations, mais consomme plus d'énergie.

\vspace{0.1cm}

L'amplitude. Selon le type de signal l'attribut possède différentes fonctions. Dans le cas d'un signal analogique l'amplitude détermine la magnitude de l'onde pour n'importe quel point dans le temps. Dans un signal numérique l'amplitude est interprétée différemment. Les signaux numériques sont encodés avec des valeurs discrètes, où chaque valeurs repésente un niveau (par exemple 0 ou 1). L'amplitude permet de faire la disctinction entre ses niveaux.

\vspace{0.1cm}

la puissance, mesurée en Watt (W). C'est la force du signal, un attribut important pour la réception du signal notamment. Bien que le Watt soit utilisé pour décrire la puissance à l'émission ou la réception, les variations de puissances sont généralement exprimées en décibels (dB). Le décibel est une unité logarithmique permettant de mesurer plus facilement les realtions entre les différents niveaux de puissance.

\vspace{0.1cm}

le \textit{Signal to Noise Ratio} (SNR). Cet attribut mesure la qualité du signal. une valeur élevée indique que le pourcentage de bruit est faible.

\vspace{0.1cm}

le \textit{Bit rate}, ou le taux de transmission mesure la quantité de donnée transmise en bit par seconde. Cet attribut est exclusif aux signaux numériques. On parle de \textit{Baud rate} pour mesurer la quantité de symboles transmise par seconde. Ce n'est pas excatement l'équivalent du bit rate car un symbole peut contenit plusieurs bits, mais le Baud rate est utilisé pour les signaux numériques et analogiques.



\section{Traitement du signal}

\subsection{Modulation}\label{mod}

La réception d'un signal radio nécessite une antenne dont les dimensions dépendent de la longueur d'onde du signal. La longueur d'onde d'un signal représente la distance entre deux points consécutifs de même phase dans l'onde. La longueur d'onde s'obtient par la formule suivante :

\begin{equation}\label{eq1}
c = f * \lambda
\end{equation}

où $c$ est la vitesse de la lumière,

$f$ est la fréquence,

$\lambda$ est la longueur d'onde.

\vspace{0.1cm}

Il est donc possible d'adapter les charactéristiques d'un signal pour le rendre compatible à différentes antennes, via la modulation. La modulation est le procédé par lequel un ou plusieurs attributs du \textit{baseband signal} (le signal modulant, contenant l'information à transmettre) vont être altéré par le \textit{carrier signal} (un signal porteur, utilisé pour être combiné avec le signal modulant) pour devenir un signal modulé, le \textit{modulated signal}.

En plus de sa compatiblité, un signal modulé a l'avantage d'être facilement transmissible sur une grand portée sans perdre en puissance.

\vspace{0.1cm}

Parmis ces différents attributs, certains sont utilisés pour effectuer une modulation. Les trois modulations les plus utilisés sont basées sur les attributs de la fréquence, l'amplitude et la phase. La modulation en fréquence (\textit{Frequency modulation, FM}) consiste à encoder l'information en faisant varier la fréquence en maitenant l'amplitude constante. La modulation en amplitude (\textit{Amplitude modulation,AM}) est le procédé inverse, c'est à dire encoder l'information en faisant varier l'amplitude tout en gardant la fréquence constante. La modulation en phase (\textit{Phase Modulation, PM)} fait varier la phase de la porteuse proportionellement à l'amplitude instantannée du baseband signal.

\vspace{0.1cm}

La modulation en amplitude est plus ancienne et est encore utilisée dans beaucoup de systèmes. Cette technique possède moins de contrainte et est notamment plus simple à implémenter.

\vspace{0.1cm}

Soient $u(t)$ un baseband signal et $v(t)$ un carrier signal, la modulation en amplitude s'effectue en multipliant les deux signaux pour obtenir le signal modulé 

\begin{equation}\label{eq2}
s_{am}(t) = u(t) . v(t)
\end{equation}

Prenons par exemple 

$u(t)$ = $sin(2\pi f_{u}t)$ avec $f_{u}t$ = 5 Hz,

$v(t)$ = $cos(2\pi f_{c}t)$ où $f_{c}t$ = 50 Hz.

La Figure \ref{term1} montre le signal modulé $s(t)$ via la modulation en amplitude.

\newpage

\begin{figure}[h]
\centering

\includegraphics[scale=0.5]{images/AM_mod.PNG}
\caption{Exemple de modulation en amplitude}\label{term1}
\end{figure}


La modulation en fréquence encode les informations dans les caractéristiques temporelles du signal transmis, ce qui la rend plus robuste aux interférence lié liés à l'amplitude qur la modulation AM. La fréquence d'un signal ne peut pas être modifiée par le bruit ou la distorsion. Cepdendant, d'autres types de distortions comme le \textit{frequency drift} (un changement non désiré de la fréquence dans le temps) peuvent affecter la qualité d'un signal modulé en fréquence.

\vspace{0.1cm}

Soient $u(t)$ un baseband signal et $v(t)$ un carrier signal, le signal modulé en fréquence $s_{fm}(t)$ est le résultat suivant :

\begin{align}
    u(t) &= \sin(2\pi f_{u}t) \\
    v(t) &= \cos(2\pi f_{c}t + \phi_{c}) \\
    s_{fm}(t) &= \cos\left(2\pi f_{c}t + \Delta f \cdot u(t) + \phi_{c}\right)
\end{align}

\vspace{0.1cm}

Reprennons le même signal en bande de base 
$u(t)$ = $sin(2\pi f_{u}t)$ avec $f_{u}$ = 5 Hz et la même porteuse 
$v(t)$ = $cos(2\pi f_{c}t + \phi_{c})$ où $f_{c}$ = 50 Hz.

\vspace{0.1cm}

La Figure \ref{term2} montre le signal modulé $s_{fm}(t)$ via la modulation en fréquence pour une phase initale du carrier signal $\phi_{c}$ = 0 avec une dérivation en fréquence $\Delta f$ = 5Hz. Attention, pour aligner le signal en bande de base à celui modulé, celui ci a subi un décalage de phase de 90°. La fonction sinus et cosinus sont liées en trigonométrie par un décalage de 90° (ou $\frac{\pi}{2} $radian) car $\sin(\theta) = \cos(\theta-\frac{\pi}{2})$

\newpage

\begin{figure}[h]
\centering

\includegraphics[scale=0.5]{images/FM_mod.PNG}
\caption{Exemple de modulation en fréquence}\label{term2}
\end{figure}

La modulation en phase permet généralement d'obtenir une meilleur utilisation de la bande passante que les autres modulations car les variations de phase peuvent encoder plus d'informations, ce qui augmente la quantité de données transmises. La modulation en phase est souvent associée à celle en fréquence. En effet les concpets sont liées par leur dépendance à la porteuse. Peu importe l'argument de la fonction de modulation choisi, il n'est plus une fonction linéaire du temps. La modulation en fréquence s'accompagne nécessairement d'une modulation de phase (présence du $\phi_{c}$). 

Soient $u(t)$ un baseband signal et $v(t)$ un carrier signal, le signal modulé en phase $s_{pm}(t)$ est le résultat suivant :

\begin{align}
    u(t) &= \sin(2\pi f_{u}t) \\
    v(t) &= \cos(2\pi f_{c}t + \phi_{c}) \\
    s_{pm}(t) &= \cos\left(2\pi f_{c}t + K_{p} \cdot u(t)\right)
\end{align}

\vspace{0.1cm}

Prenons par exemple

\vspace{0.1cm}

$u(t)$ = $sin(2\pi f_{u}t)$ avec $f_{u}t$ = 5 Hz,

$v(t)$ = $cos(2\pi f_{c}t + \phi_{c})$ où $f_{c}t$ = 50 Hz.

\vspace{0.1cm}

La Figure \ref{term3} montre le signal modulé $s_{pm}(t)$ en phase pour une phase initale ddu carrier signal $\phi_{c}$ = 0 avec un index de modulation de phase $K_{p}$ = 8.

\newpage

\begin{figure}[h]
\centering

\includegraphics[scale=0.5]{images/PM_mod.PNG}
\caption{Exemple de modulation en phase}\label{term3}
\end{figure}

On constate que pour un signal sinusoidale il est assez difficile de différencier la modulation en fréquence de celle en phase. La différence a lieue à l'emplacement des variations de la vitesse d'oscillation. Pour la modulation en fréquence, on constante que l'oscillation accélère jusqu'à atteindre le maximum quand l'amplitude du signal en bande de base est maximale, et inversément quand l'amplitude est au minimum. Les concpets de modulation de phase et de frénquences sont généralement regroupés sous le terme de modulation angulaire.


\subsection{Gestion du bruit}

L'un des attributs cités concerne le bruit. Un signal est toujours affecté de petites fluctuations plus ou moins importantes, et dont les origines peuvent être diverses. Ces perturbations, appelée bruit ou \textit{noise} en télécommunication se définissent par l'altération non souhaitée de l'intégrité d'un signal. Le bruit peut prendre différentes formes, des perturbations essentiellement impulsionnelles engendrées par des commutations de courants ou alors du bruit de fond généré dans les câbles et les composants électroniques en raison
des mécanismes statistiques de la conduction électrique. Il est possible de réduire voir éliminer l'influence des perturbations impulsionelles. En revanche, le bruit de fond est lui irreductible. Tout signal sans bruit n'existe pas, même à l'émission. Il est cependant possible que le bruit devienne invisible si son niveau est très faible. L'attribut SNR est donc un critère de la qualité du signal.


\subsection{Transformée de Fourier}

Pour effectuer une analyse de signal, sa représentation est capitale. Les Figure 1 et 2 représentent des signaux en fonction du temps écoulé, soit dans le domaine temporel. Il est possible de représenter des signaux selon une autre composante, la fréquence, c'est à dire dans le domaine fréquenciel.

\vspace{0.1cm}

La transformée de Fourier est un outil fondamental utilisé pour analyser et décomposer des signaux complexes en composantes fréquentielles. En transformant un signal dans le domaine temporel en sa représentation dans le domaine fréquentiel, la transformée de Fourier révèle les différentes composantes fréquentielles présentes dans le signal.

\vspace{0.1cm}

Pour les signaux continus, la \textit{CFT} (Transformée de Fourier continue ou juste Tranformée de Fourier) convertit une fonction du temps en fonction de la fréquence en intégrant le signal par rapport aux sinusoïdes de toutes les fréquences possibles. Cette transformation fournit les informations d'amplitude et de phase pour chaque composante de fréquence présente dans le signal. La Transformée de Fourier Continue peut être calculée de la manière suivante :  

\begin{equation}\label{eq11}
G(\omega) = \int_{-\infty}^{\infty} g(t)e^{-j\omega t} dt
\end{equation}

où : $G(\omega)$ est la Transformée de Fourier du signal $g(t)$. $\omega$ représente la fréquence angulaire $(2 \pi f)$.
Il est également possible de revenir dans le domaine temporelle, grâce à la Transformée de Fourier inverse (IFT) :

\begin{equation}\label{eq12}
g(t) = \int_{-\infty}^{\infty} G(f)e^{j2\pi ft} df
\end{equation}
Les deux équations \ref{eq11} et \ref{eq12} sont le complexe conjugué l'une de d'autre.

\vspace{0.1cm}

Reprenons la modulation en amplitude de la section \ref{mod}. En utilisant la CFT sur les trois signaux (bande de base, porteuse et modulé) de la figure \ref{term1}, la figure montre leurs CFT respectives.

\begin{figure}[h]
\centering

\includegraphics[scale=0.5]{images/CFT.PNG}
\caption{Exemple de CFT}\label{term8}
\end{figure}

La première chose que l'on constate, c'est que pour le singal en bande de base et la porteuse, on observe deux pics. Ces pics correspondent à la fréquence des différents signaux (pour rappel $f_u$ = 5Hz et $f_v$ = 50Hz) mais aussi à leur fréquence négatives. Cette duplication est due au fait que la Tranformée de Fourier produit un spectre symétrique par rapport à l'origine (dans notre cas, le centre où la fréquence vaut 0Hz). La CFT d'un signal réel possède deux valeurs (positives et négatives) pour chaques composantes du signal. On constate également que sur la CFT du signal modulé il y a 6 pics dont 4 qui ont la moitié de la magnitude des deux autres. Ces pics sont expliqués par un décalage en fréquence du à la modulation. En effet le troisième signal est un produit de deux signaux, dans le domaine fréquenciel ce produit apparait comme un décalage de la fréquence du signal en bande de base de part et d'autre de la fréquence du signal de la porteuse. C'est le théorème de la modulation. Pour un signal $u(t)$ et $v(t)$ ayant pour CFT respective $U(f)$ et $V(f)$, alors leur relation peut être exprimée de la manière suivante :

\begin{equation}
u(t)v(t) = \int_{-\infty}^{\infty} U(f)V(t - f) df
\end{equation}

Ainsi, par l'identité trigonométrique, 

\begin{equation}
sin(a)cos(b) = \frac{1}{2} sin(a+b) + sin(a-b)
\end{equation}

Le signal modulé possède un pic à 50Hz selon la porteuse, mais également la moitié d'un pics à 50 - 5 Hz et à 50 + 5 Hz selon le signal en bande de base. Comme le signal est réel, les mêmes pics sont observés à l'opposé du centre de la symétrie.

\vspace{0.1cm}

Pour les signaux discrets et échantillonnés, la \textit{DFT} (Transformée de Fourier discrète) calcule un ensemble fini de composantes de fréquence. Il est calculé à l’aide d’un nombre fini d’échantillons, ce qui donne des composantes de fréquence discrètes. Il existe un méthode optimisée pour les signaux discrets appelé \textit{FFT (Fast Fourier Transform)}\cite{fft}. Il s'agit d'un moyen plus rapide et moins couteux de calculer la transformée de Fourier, en particulier pour les signaux numériques comportant un grand nombre de points de données. La DFT calcule la transformé de Fourier pour une séquence de N échantillons en $O(N^2$) tandis que la FFT optimise le temps de calcul en $O(N log N)$ pour la même séquence. Il existe diverses variante de la FFT (par example le Cooley-Tukey FFT\cite{fft1}) ce qui peut varier les performances dépendant de l'algorithme utilisé. Les logiciels utilisés pour l'analyse de signaux dans la section \ref{fft} utilisent des algorithmes de FFT pour leur affichage dans le domaine fréquenciel.

\section{LoRa}

\textit{LoRa} (Long Range) est une technologie de communication sans fil qui permet de transmettre des données sur de longues distances avec une faible consommation d'énergie. Elle a été développée par la société française Cycleo et est maintenant gérée par la fondation LoRa Alliance, qui regroupe plusieurs entreprises et organisations du monde entier.

\vspace{0.1cm}

LoRa est principalement utilisée dans l'IoT. Elle se distingue par sa portée étendue, qui peut atteindre plusieurs kilomètres en milieu urbain et plusieurs dizaines de kilomètres en milieu rural, ainsi que par sa faible consommation d'énergie, qui permet de prolonger la durée de vie des appareils connectés. Une longue portée avec un puissance limitée induit une plus faible bande passante que les autres technologies sans fil (le Wifi, la 4G, Bluetooth etc).

\vspace{0.1cm}

LoRa utilise une bande de fréquences qui varie selon les régions du monde où LoRa est déployée :

\vspace{0.1cm}

\begin{itemize}
\item en Europe, la bande de fréquences autorisée est comprise entre 863 et 870 MHz, ce qui correspond à l'\textit{ISM radio band}, un bande dédié aux recherches qui ne nécessite pas de license d'émission.
\item aux États-Unis, elle se situe entre 902 et 928 MHz,
\item en Chine, la fréquence autorisée varie entre 779 et 787 MHz,
\item les régions restantes ont elles aussi une fourchette unique.
\end{itemize}

\vspace{0.1cm}

La technologie LoRa utilise la modulation appelé \textit{Chirp Spread Spectrum Modulation} (CSS). La modulation CSS utilise un signal chirp, c'est à dire un signal modulé en fréquence linéaire. Ce signal a une amplitude constante mais balaie tout le spectre de la bande passante de manière linéaire dans une période de temps définie. Cette technique de modulation sera détaillée à la section \ref{css}

\vspace{0.1cm}

La technologie LoRa utilise également une technique de multiplexage en temps partagé (\textit{Time Division Multiple Access}) pour permettre à plusieurs appareils de partager la même bande de fréquences de manière à maximiser l'utilisation de la capacité de transmission. Elle utilise également une technique de diffusion de données (\textit{multicast}) pour envoyer les mêmes données à plusieurs appareils simultanément, ce qui permet de réaliser des économies de bande passante et d'énergie (source : \href{https://resources.lora-alliance.org/technical-trainings/lorawan-device-to-device-multicast-communications}{LoRa Alliance})

\vspace{0.1cm}

En plus de sa portée étendue et de sa faible consommation d'énergie, LoRa se distingue par sa sécurité de transmission, qui est assurée grâce à l'utilisation de codes de sécurité uniques et à la possibilité de chiffrer les données transmises. Lora n'est pas exclusivement lié au protocole LoraWan. Ce protocol sera décrit en détails à la section \ref{lorawan}. Si LoRa opère à un niveau plus bas que la plupart des protocoles réseau, LoRaWan via son infrastructure (notament les \textit{gateway} permet entre autre aux appareils LoRa de pouvoir utiliser différents protocoles et d'être compatibles avec un grand nombre de protocoles de communications comme \textit{TCP/IP (transport layer protocol), HTTP (hypertext transfer protocol ou MQTT(message queuing telmetry transport)}.

\vspace{0.1cm}

Toutes ces  particularités font de LoRa une technologie complémentaire à celles déja existantes plutot que rivale.
LoRa se compose de deux éléments principaux : la couche physique de la technologie et LoRaWAN, la couche MAC (\textit{Media Access Control}, une sous couche de la couche liaison de données dans le modèle OSI \textit{Open Systems Interconnection}. la couche physique de LoRa gère la fréquence radio ainsi que la modulation. LoRaWAN gère les aspects réseaux comme la sécurité, la propagation, l'adressage et la sécurité.

\subsection{couche physique LoRa}

\subsubsection{Découpage de la couche physique}

\begin{figure}[h]
\centering

\includegraphics[scale=1]{images/physical_lora_rx.PNG}
\caption{Etapes de la transformation des données dans un émetteur LoRa\cite{loraphy}}\label{term4}
\end{figure}


Les étapes de la conception de l'envoi de données dans la couche physique de LoRa sont montrés dans la figure \ref{term4} :

\vspace{0.1cm}

\begin{itemize}
\item Le codage de canal (\textit{channel coding}) est une technique utilisée dans les systèmes de communication sans fil pour améliorer la robustesse et la fiabilité de la transmission des données. Dans le cas de LoRa, le codage de canal utilise le \textit{Forward Error Correction (FEC)} pour corriger les erreurs causées par du bruit. La méthode FEC ajoute de l'information redondante sur les données.
\item Le mélange de canal (\textit{channel interleaving}) suit le codage de canal.  
Cette technique consiste à réarranger les bits ou les symboles de données en les dispersant sur plusieurs canaux (ici on fait référence à des streams ou a des bandes de fréquences spécifiques plutôt qu'à des canaux physiques). Cela permet de réduire l'impact de \textit{burst errors}, des erreurs consécutives.
\item Le blanchiment de canal (\textit{channel withening)} est la dernière étape avant la modulation du signal.
Cette technique consiste à utiliser une transformation aléatoire ou pseudo-aléatoire des données avant de les transmettre, de manière à répartir le spectre des fréquences de la transmission sur une large gamme de fréquences. C'est une technique mathématique qui consiste a effectuer une transformation linéaire des données avec un matrice de covariance en un nouveau set de données dans la covariance est la matrice d'identitée. Le du blanchiment est de réduire la corrélation entre les différentes composante fréquencielles et assurer que le signal possède une puissance similaire tout le long de son spectre.
\item La modulation CSS est l'étape principale de LoRa. En effet, les étapes précédentes sont communes à de nombreuses technologies, mais la particularité de LoRa provient de la modulation. Cette étape est détaillée dans la section \ref{css}.
\end{itemize}

\vspace{0.1cm}

Chacune des étapes décrites doit être inversément réalisée pour le récepteur. Ainsi pour la récupération de donnée à l'arrivée, l'appareil récepteur gère la démodulation, le déblanchiment, le démellement et de décodage.

\vspace{0.1cm}

Cette analyse a été faite en \textit{Reverse Engeneering} par Alexandre Marquet, Nicolas Montavont et Georgios Z. Papadopoulos \cite{lorareverse}. Le reverse engineering consiste à analyser un produit ou un système afin de comprendre comment il fonctionne ou d'identifier ses principes de conception. Dans le contexte de LoRa, le reverse engineering examine la technologie derrière LoRa afin de comprendre ses principes de base et sa conception.


\subsubsection{Modulation CSS}\label{css}

Contrairement aux modulation classique en amplitude ou en fréquence, la modulation CSS étale le signal sur une large bande de fréquence. La modulation en fréquence est linéaire et utilise des chirps. un chirp est un signal dont la fréquence change en continue tout en conservant une amlitude constante. Il existe deux types de chirps : les $upchirp$ et $downchirp$.
Dans un upchirp la fréquence augmente avec le temps tandis que dans un downchirp la fréquence diminue. Soit $s_{chirp}(t)$ un signal chirp avec

\begin{equation}\label{eq3}
s_{chirp}(t) = sin(2\pi(f_0 + (\frac{f_1 - f_0}{T})t)t)
\end{equation}

alors la figure \ref{term5} montre $s_{chirp}$ en fonction du temps où $f_0$ = 10Hz, $f_1$ = 100Hz et $T$ = 1 seconde. On observe que le signal oscille de plus en vite plus vite au fur et à mesure que le temps augmente.

\begin{figure}[h]
\centering

\includegraphics[scale=0.18]{images/CSSupchirp.png}
\caption{Example d'un signal modulé en upchirp en fonction du temps}\label{term5}
\end{figure}

\vspace{0.1cm}

Le signal est donc séparé sur une large bande de fréquence, permettant par exemple plusieurs transmissions sans causer d'interférence.
La modulation CSS est l'une des principales contributions au fait que LoRa possède une faible consommation et une longue portée. Cette technioque est très bien intégrée aux appareils a faible puissance utilisé par LoRa.

\subsubsection{Spreading factor}

LoRa permet d'envoyer des paquets sur une longue distance à faible puissance. Selon l'environement dans lequel les appareils LoRa sont présents, il peut être utile de pouvoir ajuster certaines capacités.

\vspace{0.1cm}

Le facteur d'étalement (\textit{Spreading Factor, SF}) permet de déterminer le taux de variation de fréquence pour un signal. Modifier le spreading factor ajuste différentes propriétés de la communications \cite{thethingsnetworkSF}. Par exemple, si on augmente le spreading factor, les quatre conséquences principales sont :

\vspace{0.1cm}

\begin{itemize}
\item l'augmentation de la portée. En effet augmenter le SF réduit le bitrate et augmente le \textit{processing gain} (l'augmentation de la puissance du signal atteint en l'étalant sur une plus large bande).
\item Augmentation de la résistance aux interférences. Comme le signal est étalé sur une bande plus largeur, il y a moins de risque de subit des interférences.
\item Plus petit débit de données. Le spreading factor controle le taux de chirp, et du coup la vitesse de transmission de donnée. Augmenter le speading factor signifie ralentir la vitesse d'émission des chirps. Pour chaque augmentation du spreading factor, le taux de transmission de donnée est réduit de moitié.
\item Plus faible consommation. Les données transmises à un taux plus faible consomment moins d'énergie, ce qui prolonge la durée de vie des appareils dont l'économie d'énergie est une priorité.
\end{itemize}

Diminuer le spreading factor engendre l'effet inverse \cite{thethingsnetworkSF}.

\newpage

\subsubsection{Structure d'un paquet LoRa}\label{packetlora}

\begin{figure}[h]
\centering

\includegraphics[scale=0.4]{images/lorapacket.png}
\caption{Structure d'un paquet Lora\cite{lorapacket}}\label{term6}
\end{figure}


La figure \ref{term6} montre la structure d'un paquet Lora. Un paquet LoRa contient 3 parties différentes \cite{loraphy} :

\vspace{0.1cm}

\begin{itemize}
\item Le \textit{preamble}. La première partie du paquet, composée d'un nombre variable d'upchirps. La valeur par défaut est fixée à 8 upchirps minimum. L'émetteur radio ajoute à cela un peu plus de 4 symboles (4.25), qui contiennet l'identificateur réseau ainsi que deux downchirps de synchronisation de fréquence. Ceci fixe le préambule à 12 symboles.
\item Le \textit{header} du paquet. Les informations sur la taille du paquet, le code rate, la présence d'un CRC (\textit{cyclique redundancy check}) et la checksum sont incluses dans l'en-tête.
\item Le \textit{payload}. La dernière partie du paquet qui contient les données à transmettre. La taille maximale du payload est de 255 octets. En plus des données, le payload peut également contenir un CRC pour la détection d'erreurs. La longueur du CRC est généralement de 16 bits.
\end{itemize}


\subsection{LoRaWAN}\label{lorawan}

LoRaWAN est un protocol de type \textit{Low Power Wide Area Network (LPWAN)} désigné pour la communication longue portée. Ce protocole opère avec la technologie LoRa et lui fournit une infrastructure capable de maintenir une communication à longue portée et à faible cout dans l'IoT.

\subsubsection{Aspets généraux de la technologie}

LoraWan bénéficie donc d'une faible puissance de consommation et d'une portée accrue. Elle est également efficace dans différents environements. Le signal est capable de pénétrer diverses terrains et structures.
Le déploiment d'une infrastructure LoRa ne nécessite pas de license, et son réseau peut être public ou privé. Les caractéristiques générales de LoraWan sont disponibles sur le site \href{https://www.thethingsnetwork.org/docs/lorawan/}{The Thing Network}.

\vspace{0.1cm}

Le coeur de LoRaWAN réside sur la gestion de l'énergie, permettant aux appareils de fonctionner avec une consommation d'énergie minimale, prolongeant leur durée de vie tout en garantissant une fonctionnalité à long terme. A cette caractéristique de faible consommation d'énergie s'ajoute ses capacités en termes de portée, capable de pénéter diverses environements. Cela rend la technologie efficace aussi bien milieu rurale qu'urbain. LoRaWAN opère sur une bande de fréquence qui ne nécessite pas de license d'émition, par exemple sur la bande ISM pour \textit{Industrial, Scientific, and Medical}. Les bandes ISM, (868 MHz en Europe ou 915 MHz aux USA) sont disponibles pour l'utilisation de différentes technologies, incluant LoraWAN.

\vspace{0.1cm}

LoRaWAN possède des capacités de géolocalisation, permettant au réseau de détecter et de localiser précisément les appareils au sein de son domaine. LoraWAn utilise différentes méthodes pour localiser ses appareils comme \textit{Received signal strengh indication (RSSI)}, \textit{Time difference on Arrival (TDOA)}, une triangulation ou alors une combinaison de plusieurs des méthodes. Certaines de ses méthodes seront détaillées dans la section \ref{identification}

\vspace{0.1cm}

LoRaWAN utilise des protocoles de sécurité \textit{end-to-end}, aussi bien dans un réseau public intégré que dans un réseau privé.L'architecture LoraWan (décrite en détails dans la section \ref{topolora}) contient plusieurs couches de sécurité. Au niveau des \textit{end devices}, une routine d'identification est imposée avant l'accès au réseau. seul les appareils de confiance son donc autorisé à communiquer. Ensuite, une fois la communication commencée, les données sont chiffrées avant d'être transmise dans le réseau. Le framework sécuritaire de Lora ne se limite pas à l'autentification et au chiffrage. LoraWan gère également les mise à jour en continue par les airs, ainsi qu'une supervision continue sur d'éventuelles intrusions.

\vspace{0.1cm}

Avec toutes ces caractéristiques, LoraWan s'est développé dans de nombreux domaines aussi bien environementaux qu'industriels. Les principales utilisations de LoraWan actuelles sont les suivantes (toutes les applications \href{https://www.semtech.com/lora/lora-applications}{ici}):

\vspace{0.1cm}

\begin{itemize}

\item la surveillance environementale en général \cite{lorauc1}. Lorawan peut être déployé pour surveiller des niveaux de températures, d'humidité, de bruits ou encore d'autres paramètres dans n'importe quel milieu. Une compagnie Hollandaise, \href{https://www.sensoterra.com/technology/global-lorawan-networks/}{Sensoterra}, utilise notamment LoraWan pour surveiller la qualité des sols.
\item Les \textit{smart cities} \cite{lorauc2}. LoraWan est actif sur différents aspects comme la gestion intelligent de l'éclairage, la gestion des déchets, la surveillance, etc.
\item l'embarqué industriel \cite{lorauc3}. La maintenance et la surveillance de matériel et de l'équipement peut être gérée par Lorawan. \href{https://consulting.tatasteel.com/our_expertise/plant-infrastructure-and-logistics/}{TataSteel}, une compagnie indienne, utilise LoraWan pour ces équipements industriels.
\item la prévention de catastrophe naturelle. Que ce soit en prévision\cite{lorauc41} ou après\cite{lorauc43} d'éventuelles catastrophes naturelles, la longue portée et la surveillance en temps réel sont des atoux cruciaux pour ce genre d'évènement.
\end{itemize}

\vspace{0.1cm}

Cependant, toutes ses caractéristiques entrainent un certain nombre de limitations. La restriction de la fréquence en fonction de la région peut rendre le déploiment d'une même infrastructure à différent endroit dans le monde plus difficile. Cela peut aussi entrainer des problèmes de compatibilité entre régions, notament pour des chaines logistique ou d'approvisionement qui en traverse plusieurs.

\vspace{0.1cm}

Une faible consommation de puissance avec une grande portée a un impact sur la taille et la vitesse de l'information. La taille du payload d'un message est limitée entre 51 et 241 octets. La vitesse de transmission est également peu élevée, atteignant un maximum de 5.5kbps sur une largeur de bande de 125kHz.

\vspace{0.1cm}

La communication au sein d'un réseau LoraWan se fait en grande partie de manière asynchrone. La synchronisation dépend de la classe de l'appareils, qui est détaillé dans la section \ref{topolora}. C'est un avantage pour maintenir une grande autonomie de batterie pour les appareils. LoraWan possède un système pour limiter les colisions entre messages si plusieurs appareils communiquent simultanénent. Ce système est basé sur une combinaison entre \textit{Listen before talk (LBT)} et des delais aléatoires\cite{loracolision}. Il est néamoins possible que dans un environement très dense des collisions puissent encore se produire. La comunications asynchrone et le système d'évitement de collision entraine une augmentation du temps entre les envois et la réception de message.

\subsubsection{Topologie de Lorawan}\label{topolora}

\begin{figure}[h]
\centering

\includegraphics[scale=0.1]{images/architecture.png}
\caption{Topologie de l'infrastructure LoraWan (source : \href{https://www.thethingsnetwork.org/docs/lorawan/architecture/}{The Thing Newtork})}\label{term7}
\end{figure}

La figure \ref{term7} montre les 4 types d'appareils qui composent la topologie d'un infrastrucure LoraWan.
Les end devices sont les noeuds qui collectent les informations à envoyer à travers le réseau. Ils sont catégorisés en trois sous classes : A, B et C. Les apareils de classe A sont les plus économes en énergie. Ils ont été créés pour conserver leur énergie et communiquent exclusivement en comunication asynchrone. Les appareils de classe A écoutent les messages provenant des serveurs uniquement après avoir eux-même transmis un message. la classe A regroupe les appareils les moins énergivores. Les appareils de classe B sont assez similaires à ceux de classe A, mais sont occasionellement synchronisés avec les serveurs du réseau. Ils possèdent des capacités supérieurs de reception leur permettant de se synchroniser avec le scheduler des serveurs, ce qui augmente considérablement l'efficacité du temps de réponses dans le réseau. Finalement, les appareils de classe C sont en écoute permanente de messages provenant des serveurs. Ils sont les plus réactifs mais également les plus énergivores. Les end devices sont donc classés selon deux paramètres : leur réactivité et leur consomation d'énergie. En fonction de leur classe, ils ont également la possibilité de recevoir des messages server après avoir transmis de l'information. L'envoie d'un message d'un end devices vers les serveurs est appelé \textit{uplink message} et l'envoie d'un message depuis les serveur vers les end devices est appelée \textit{downlink message}.

\vspace{0.1cm}

Les gateways jouent le role d'intermédiaire entre les end devices et le serveur réseau. Ils recoivent les transmissions depuis les end devices dans leur zone de couverture
et forward les messages vers le serveur réseau. Les gateways peuvent écouter plusieurs fréquence simultanénent (\textit{multichanneling}) là où les end devices n'écoutent qu'une seule fréquence. Les gateway gèrent la communication radio avec les end devices en utilisation la modulation de LoRa.

\vspace{0.1cm}

Le serveur réseau est la composante centrale de l'infrastructure. Il gère tout le réseau, que ce soit les données reçues des gateays, l'identification et l'activation des end devices dans le réseau, le routing ou encore l'adapation du data rate. Le serveur réseau supervise également l'aspect sécurité au sein du réseau en gèrant les clés de chiffrage et les protocoles de sécurité.

\vspace{0.1cm}

Le serveur application de LoraWan reçoit les données forwardées depuis le serveur réseau. C'est l'interface entre le réseau de LoraWan et les différents services ou applications d'utilisaturs finaux. Les utilisateurs intéragissent avec le serveur d'application pour n'importe quelle action a effectuer sur le réseau ou pour la récupération de données du réseau. Les données reçues par le serveur réseau sont traduites par le server d'application avant d'être interprétée par l'utilisateur final.

\subsubsection{Sécurité}

La sécurité dans l'architecture LoraWan se concentre sur trois axe principaux :

\begin{itemize}
\item l'authentification : qui communique avec qui.
\item L'intégrité : les données ne sont pas altérées entre l'émetteur et le recepteur.
\item La confidentialité : les données ne sont visible par personne au sein du réseau hormi l'émetteur et le récepteur. 
\end{itemize}

\vspace{0.1cm}

La sécurité repose sur le chiffrage des données. Les données sont chiffrées en utilisant l'algorithme de cryptographie AES (\textit{Advanced Encryption Standard}). La taille des clés est de 128 bits. Ce choix est motivé par un équilibre entre une sécurité suffisante et une consommation réduite des ressources\cite{loraes}.

\vspace{0.1cm}

Il y a deux types de clés utilisées dans LoraWan. La \textit{root key} est la clé partagée entre un end device et le serveur réseau. Cette clé est utilisée pour l'authentification initiale et l'établissement d'une communication entre deux éléments du réseau. Cette clé n'est jamais transmise par les air, elle est stockée dans un \textit{join server}. Un join server est un server dédié au contenu sensible à l'activation du matériel dans un réseau LoRaWAN. Il autentifie le réseau et les application du servers. Il gère les root keys. il génère également le second type de clés de LoraWan, les \textit{session keys}.

Les session keys sont des clés générées dynamiquement par le join server et utilisées durant l'échange de données pendant une session. Il y a deux session keys différente, la \textit{AppSKEY} pour le chiffrage des payloads d'application, et la \textit{nwkSKEY} pour les fonctionalités du réseau (le chifrage à la couche MAC, les vérification d'integrité, etc).

\subsubsection{Session}

L'établissement d'une session entre un end devices et le réseau LoraWan peut se faire de deux façons différentes.

La première méthode est une méthode dynamique appelée \textit{Over the Air Activation(  OTAA)} et se déroule de la façon suivante: 
\begin{itemize}
\item Le device possède initialement deux indentificateurs, un DevEUI et un appEUI.
\item La requête pour rejoindre le réseau est initiée par le end device. Il en envoir un message \textit{join request} au serveur réseau. La join request contient ses identificateurs, ainsi qu'un nombre aléatoire généré par le device.
\item Le serveur réseau accepte (ou décline) la requête et vérifie les identifiateurs du device dans ses enregistrements. 
\item Si la requête est acceptée, le server génère ensuite un nombre aléatoire appelé \textit{DevNonce} et renvoie un message \textit{join accept} contenant le DevNonce, l'adresse du device ainsi que les clés (NwkSKey et appSKey) de session.
\item le end device reçoit le message join accept. Il extrait les clés envoyées et calcule ses propres clés de session avec ses paramètres (les clés envoyé par le join server ainsi que le devNonce).
\item le device fait maintenant parti du réseau. Chaque message que le device va envoyer au serveur sera chiffré avec ses clés.
\end{itemize}
        
\vspace{0.1cm}
        
La seconde méthode est hardcodée et permet à un end device de rejoindre directement le réseau sans passer par l'indentification. cette méthode est appelée \textit{activation by personalisation (ABP)}. Voici la procédure de la session :
\begin{itemize}
\item Le device possède à l'avance son adresse ainsi que ses clés de session.
\item Le device est déploiyé au préalable dans la zone de couverture du réseau LoraWan.
\item Sans devoir itinialiser de procedure \textit{join request}, le end device transmet directement ses données au serveur en utilisant ses clés préconfigurées. L'échange de clés avec le serveur n'a pas lieu.
\end{itemize}

Cette seconde procèdure a comme avantage d'être plus rapide à exécuter car toute la partie d'initialisation est passée. Le processus d'initialisation peut être contraignant en ressource ce qui rend la méthode ABP moins énergivore. Cependant l'utilisation de clés hardcodées directement dans les devices est une pratique moins sécuritaire. Comme pour la taille de clés, il y a un équilibre entre consommation d'énergie et sécurité.


\chapter{Travaux similaires et autres contributions}

trois partie,une partie générique comprenant certaines approche pour identifier des devices dans l'iot en général. Une seconde partie reprenant des approce d'analse de signaux lora. Finalement une section joigant les deux première partie pour atteindre l'objectif du travail, l'identification de devices spécifique à lora via l'analyse de leur signaux radio.


\section{identification de device dans l'iot}

avant de s'intéresser à l'identification de device lora, besoin de regarder plus largement dans l'iot et de voir si l'identification de devices est étendue à l'iot en général (oui), et comment l'indentification est acomplie.

piste : RSSI, TCP finguerprinting, taffic related patterns (behaviour, power consumption, packet timing) 

\section{analyse de signaux lora}

differente travaux sur comment interpréter un signal. des analyses des différent composantes du signal 

\section{identification de device lora}

radio frequency finguerprinting identification (RFFI). trouver des charactéristique hardware pour identifier des devices.
article sur les méthode de constellations traces
article sur les spectogrammes.

\chapter{Expérimentations}


\renewcommand{\leftmark}{EXPERIMENTATIONS}

\section{Matériel}

\subsection{radio logicielle}

La radio logicielle ($SDR$, pour $Software$-$Defined Radio$) est une technologie qui permet de mettre en œuvre des systèmes de radio à l'aide de logiciels plutôt que de matériel. 

Dans les systèmes de radio traditionnels, les différentes fonctions de la radio, comme l'accord sur une fréquence spécifique, la modulation et la démodulation du signal, et le filtrage du bruit, sont mises en œuvre à l'aide de composants matériels tels que des oscillateurs, des amplificateurs et des filtres. En revanche, les systèmes SDR utilisent des logiciels pour effectuer ces fonctions, ce qui les rends beaucoup plus flexible car chaque composante est reconfigurable. Les radios logicielle sotn capable d'opérer sur une large portée de fréquence, aussi bien très basse fréquence comme haute fréquence.
Les $SDR$ peuvent jouer le role d'éméteur ou de récepteur voir les deux.

\subsubsection{RTL SDR dvb T}

La première radio utilisée comme récepteur. possède différente composante :

rtl2832U: digitalise les signaux RF et les evnoie à l'ordinateur.
Tuner chip : le tuner permet d'ajuster la fréquence. Grace à ça la sdr peut couvrir une larger portée.
port usb : pour raccorder la sdr à l'ordinateur.

\subsubsection{RTL-SDR R820T2}



meilleure qualitée



\subsubsection{hackRf}

autre radio logicielle. plus (cher et) complète. meilleure qualité de signal que les rtl-sdr "classiques".

\subsection{Module d'émission Lora}

\subsubsection{module RN2483}

Le microchip RN2483 est un module de technologie spécifique à LoRa. Cet appareil permet de communiquer à longue portée et à faible coup grêve à l'utilisation de la modulation basé sur LoRa.

quelques spécificités du module :

technologie LoRa
faible puissance (ideale pour de l'iot car faible consommation)
fréquence à 433, 868 et 915MHz (regarder la régions adéquate)
AT command : configurable via un set de commande
compatible avec le protocole LoRaWAN pour établir ou rejoindre ce type de réseau.

fonctionne par entrée de commande (aucun retour écran donc aucune faute possible)

différentes commandes/ utilisation :

sys get ver : demande la verion du module, reçoit en réponse 
radio set (param) (value) : ajuste le paramètre pour l'adapter a la valeur souhaitée.

\subsubsection{pycom lopy}

besoin  d'un environement python , qqs configuration nécessaire.

\subsubsection{module arduino}

besoin d'un IDE arduino, possibilité de configurer une largeur de bande bien plus faible que pour les autre module, très pratique pour analyser le signal en détail.

\subsection{logiciel}

\subsubsection{gqrx}

logiciel open source d'analyse de fréquence radio pour les SDR.

installer gqrx via apt. (ubuntu)

sélectionner le périphérique pour analyse

\textcolor{red}{image choix périphérique}

visualisation du spectre

deux forme d'affichage,en spectre et en cascade.

L'affichage du spectre fournit une représentation graphique en temps réel du spectre RF sur une gamme de fréquences.
Il montre la puissance du signal de différentes fréquences sur une plage de fréquences spécifiée.
L'axe des x représente la fréquence, tandis que l'axe des y affiche la force du signal (mesurée en dB).

L'affichage en cascade est un spectrogramme qui visualise la force du signal au fil du temps.
Il montre une série d'instantanés de spectre empilés les uns sur les autres, où l'intensité de la couleur représente la force du signal.
Chaque ligne horizontale du tracé en cascade représente une vue du spectre capturée à un moment précis, créant ainsi un enregistrement historique de l'activité du signal.
L'axe vertical représente la fréquence et l'axe horizontal représente le temps.

\textcolor{red}{image affichage spectre}

configuration de la réception :

imput control (pas trop touché)

FFt settings : très important règle la ff size, le raffraichissemetn d'image. le laps de temps. l'averaging

Le paramètre Panadapter dB fait référence à l'échelle verticale dans la vue du spectre. Il représente la force du signal des fréquences radio reçues affichées sur l'axe vertical du graphique du spectre. Le réglage du paramètre Panadapter dB modifie l’échelle verticale de la force du signal affichée dans la vue du spectre.

Le paramètre Waterfall dB concerne l'intensité de la couleur ou l'ombrage des fréquences affichées dans le tracé en cascade.Le réglage du paramètre Waterfall dB modifie l'intensité utilisée pour afficher la force du signal dans le tracé en cascade, permettant ainsi d'ajuster le contraste ou la visibilité des signaux plus faibles ou plus forts.


\subsubsection{Universal radio hacker, URH}

logiciel open source. similaire à gqrx pour fonction d'analyse du signal. Il est possible de visualiser les signaux de manières analogique, démodulé, en spectrogramme ou en vue I/Q. Il est possible de découper les signaux, notamment pour supprimer les parties "vides" dans les enregistrements. Possibilité de sauvegarder des signaux enregistré dans des fichier. Urh supporte différents formats pour les signaux :

 \textit{.complex} files with complex64 samples (32 Bit float for I and Q, respectively). This is the default signal file format.
 
\textit{.complex16u} using two unsigned 8 Bit integers for I and Q

\textit{.complex16s} using two signed 8 Bit integers for I and Q

\textit{.complex32u} using two unsigned 16 Bit integers for I and Q (since v2.7)

\textit{.complex32s} using two signed 16 Bit integers for I and Q (since v2.7).

Il est également possible de lire des fichiers de données qui n'ont pas été enregistré avec URH tant qu'ils sont dans les formats supportés.

\section{Librairie python}

utilisation de python : pourquoi ?
librairie très utile dans le domaine comme NumPy, Pandas, Scikit-learn, PyCM et FiPy.
Librairie disponible pour les radio logiciel hackrf et rtlSDR.
Librairie compatible avec la sauvegarde et l'utilisation des données. Bonne documentation notament sur les formats des nombres complexes.
Capacités pour le machine learning. Utilisation de diverses algorithmes comme kmeans déja implémenté dans des librairies.
Intégration avec des modules, la librairie fipy gère l'un es modules utilisé pour les éxpérimentations.

numpy : pour complexe conjugué
matplotlib : pour les plot des diagrammes
datashader : pour la coloration du diagramme de constellations


\section{Géneration et réception d'un signal LoRa}

dans un premier temps les signaux sont généré manuellement sans automatisation, le but étant de reconnaitre et d'analyser la structure d'un signal Lora. le premier signal est généra via le module rn2483. La documentation est disponible au lien suivant (lien). Via python, il est possible d'utiliser la librairie \textit{Serial} pour se connecter au port usb reliant le module à l'ordinateur. Ensuite, via les différentes commandes, on configure le module en modifiant les paramètres suivant : 

\begin{itemize}
\item la modulation, en LoRa.
\item la fréquence. 868MHz, la fréquence de la bande ISM , la bande d'émission en Europe.
\item la largeur de bande. Dans un premier but de visualiation, on souhaiterai que le message soit le plus long possible dans le temps pour pouvir l'observer facilement. Ainsi la largeur de bande choisie est de 125Khz, ce qui est la plus petite possible valeur que permet le module.
\item la puissance du module, au maximum (14W).
\item le spreading factor. Même raisonement que pour la largeur de bande, on souhaite que le \textit{time on air} soit le plus long possible, donc la valeur la plus grande possible est choisie (SF = 12)
\item le coding rate. Il y a 4 valeur possible:s 4/5, 4/6, 4/7 and 4/8. Cela signifie que tout les 4 bits seront codé par 4, 5, 6, 7 ou 8 bits de transmissions en fonction de cette valeur. Plus la valeur st faible (la plus faible étant 4/8), plus le time on air sera élevé, car cela prend plus de temps pour transmettre le message.
\end{itemize}

Les commandes relatives à la configuration sont disponibles en annexes.
Une fois que le module est configurer, il faut paramètrer le recpeteur, la radio logicielle. La librairie pyrtlsdr permet de pouvoir configurer le récepteur, elle sera utlisée pour les expérmientations dans la section 3.4. Dans un premier temps, les logiciels comme gqrx ou urh permettent également de pouvoir configurer le récepteur. Leur utilisation a déja été décrite dans la section 3.1.3. Les paramètres à configurer sont les suivant : 

\begin{itemize}
\item la fréquence. La fréquece d'écoute. Idéalement la même fréquence de celle de l'émetteur (868MHz). Cependant celle ci sera légèrement décalée pour contrer un effet
\item le taux d'échantillonage. Il est possible de choisir la quantitté d'échantillons traitée chaque seconde. Un taux plus élevé donnera un signal plus complet. Le taux minimal ne doit jamais être inférieur à deux fois la largeur de bande. (Théorème de nyquist shannon : fe > 2fmax)
\item le gain. Dépendant de la qualité du signal il peut être nécessaire d'ajouter du gain, c'est à dire d'amplifier la force du signal. Un gain trop élevé peut saturer le signal, quand l'amplitude dépasse la portée du récepteur.
\end{itemize}

Une fois l'émetteur et le récepteur configuré, il est maintenant possible de visualiser des signaux Lora.

\subsection{analyse avec gqrx}



\subsection{analyse avec urh}

affichage du signal capturé, d'abords sous forme analogique, puis en spectrogramme.
Décomposition du signal, on observe des "chirps". En analogique, augmentation de la fréquence (unchirp) et diminution de la fréquence (downchirp). En spectrogramme, augmentation est plus visuelle encore, on voit de manière net les chirps.

Dans le signal, on reconnait donc le préambule du signal composé de 10 upchirps et 2 downchirps (selon la théorie).

Attention, la fréquence d'écoute des sdr ne doit pas être exactement à 868Mhz. En effet, voilà à quoi ressemble si la fréquence d'écout est la même que la fréquence d'émission. Il faut prendre en compte la largeur de bande du signal, dans le cas ou le singal a une largeur de bande de 125KHz, il faut décaler la fréquence d'écoute pour recentrer le signal, ainsi on évite d'avoir des fréquence qui sont interprétée comme "négatives" par URH. Dans la figure, la fréquence est décalé de 125/2 soit 867,935Mhz.

\subsection{automatisation du signal et preprocessing}

Dans un second temps, besoin d'automatiser la génération et l'enregistrement des signaux afin de pouvoir travailler avec un grand nombre d'échantillons. La librairie python pytlsdr permet de ne pas devoir passer par un logiciel comme urh pour sauvegarder les échantillons. La méthode d'enregistrement est la suivate :

\begin{itemize}
\item d'abords, configurer la module et la rtlsdr avec les différent paramètre
\item ensuite l'antenne commence a enregistrer. L'émetteur est informé que la radio logiciel est en écoute et envoie un signal. 
\item dès que la radio reçoit le signal, elle informe l'émetteur de se mettre en attente car le preprocessing commence.
\item Le signal reçu est découpé pour ne conserver que le preambule, et puis enregistrer dans un fichier.
\item dès que l'enregistrement est terminé, le cycle recommence.
\end{itemize}

la code source est disponible en annexe.


\section{Méthode "Constellation traces"}

objectif, identification du device via son frequecy offset.

selon l'article (citer article), possible de performer la méthode soit uniquement sur le  préambule, soit sur l'intégralité du singal. 

idée: le received singal contient le baseband singal ainsi qu'un rotation factor instable. pour pouvoir recover cette partie du signal, besoin d'effctuer une opération différentielle suivante : $$ x(t) . x(t+n) e -j2\pi on $$
apparation d'un nouveau rotation factor, mais stable. Besoin de trouver deux inconnue,\textit{delta f} et \textit{n}. n est le differential interval. il se calcule de la manière suivante : $$ Rs = BW / 2sf $$ 
$$ N = fs / Rs $$

delta f est la difference entre le transmitter carrier frequency et le receiver carrier frequecy.

application : récupérer les samples I/Q du signal n ayant au préalable "clean" le signal. utilisation d'un gradiant pour observer la dentsité sur le plot. noramlement des zones plus denses apparaissent.
coloration : utilisation de la librairie data shader. 


clustering, le but de conserver les parties les plus dense (95pourcent du point le plus dense) plot avec les différents appareils.
librairie panda et numpy. découpage en zones (bins?) sous forme d'une grille, calcul de nobre de points dans chaque zone, la zone avec le plus grand nombre de point sers de maximum. 


\chapter*{Conclusion}
\addcontentsline{toc}{chapter}{Conclusion}
\renewcommand{\leftmark}{CONCLUSION}

L'objectif du travail était de pouvoir identifier des noeuds de l'Internet of Things utilisant la technologie \ac{LoRa} en se basant uniquement sur les caractéristiques spécifiques de leur signal radio, différenciées par une dispersion sensible de la performance des composants électroniques intégrés dans chaque émetteur.
\vspace{0.1cm}

Cette étude, le Radio Frequency Fingerprinting Indentification a notamment été réalisée par Yu Jiang, Linning Peng, Aiqun Hu, Sheng Wang, Yi Huang et Lu Zhang dans \cite{loraDCTF} via l'utilisation de diagrammes de constellation. Ils ont montré qu'il était possible en analysant les coordonnées géographiques d'une signature radio dans une \ac{DCTF} de pouvoir retrouver l'appareil \ac{LoRa} émetteur. Ces auteurs se sont concentrés sur la fréquence de transmission unique des émetteurs \ac{LoRa}, de leur déviation spécifique vis-à-vis de la fréquence théorique, comme signature d'identification. Les diagrammes de constellation ont ainsi été générés à partir des composantes \ac{I/Q} des signaux modulés transmis par ces émetteurs selon leurs caractéristiques fréquentielles différenciées.

\vspace{0.1cm}

Afin de reproduire cette étude, il a fallu dans un premier temps apprendre en détail le fonctionnement de la technologie \ac{LoRa} et  de la modulation qu'emploient les émetteurs \ac{LoRa}. Ensuite, maitriser le fonctionnement et l'utilisation d'une radio logicielle avant de finalement commencer à capturer des signaux radio.

\vspace{0.1cm}

L'analyse des signaux a été possible grace à des logiciels \textit{open source} et flexibles comme GQRX et \ac{URH}, avant d'avoir recours à des librairies Python pour pouvoir étudier plus en profondeur les caractéristiques de ces signaux.

\vspace{0.1cm}

Finalement, la reproduction de l'étude aura révélé des résultats qui ne s'ali\-gnent pas avec les affirmations de l'article \cite{loraDCTF}. Ceux-ci montrent cependant qu'une identification basée sur le \ac{RFFI} est bien possible, non pas selon les coordonnées géographiques de la signature, mais plutôt selon sa forme géométrique.

\vspace{0.1cm}

Ainsi, une nouvelle piste dans l'identification de noeuds LoRa basé sur la méthode des diagrammes de constellation pourrait se concentrer sur la reconaissance d'images afin de retrouver, pour des paramètres d'émission fixes, une image correspondant à la signature du signal. L'utilisation des réseaux de neurones convolutifs est déja présente dans le \ac{RFFI}. Son ajout à la méthode des \ac{DCTF} pourrait permettre de facilement reconnaitre la signature.


\bibliographystyle{latex8}
\bibliography{biblio}

\newpage
\appendix

\chapter{Annexe A :Code module Arduino}\label{codearduino}


\renewcommand{\leftmark}{Annexes}

\begin{lstlisting}[caption={Example Python code}, label={lst:example}]
# This is a Python function
def my_function(x, y):
    """
    This function computes the sum of two numbers.
    
    Parameters:
    x (int): The first number.
    y (int): The second number.
    
    Returns:
    int: The sum of x and y.
    """
    return x + y

# Example usage of the function
result = my_function(3, 4)
print("The result is:", result)
\end{lstlisting}


\chapter{Annexe B :Code module RN2483}\label{codern}

\begin{lstlisting}[caption={Example Python code}, label={lst:example}]
# This is a Python function
def my_function(x, y):
    """
    This function computes the sum of two numbers.
    
    Parameters:
    x (int): The first number.
    y (int): The second number.
    
    Returns:
    int: The sum of x and y.
    """
    return x + y

# Example usage of the function
result = my_function(3, 4)
print("The result is:", result)
\end{lstlisting}

\chapter{Annexe C :Implémentation émission réception automatisé}\label{codeauto}

\begin{lstlisting}[caption={Example Python code}, label={lst:example}]
# This is a Python function
def my_function(x, y):
    """
    This function computes the sum of two numbers.
    
    Parameters:
    x (int): The first number.
    y (int): The second number.
    
    Returns:
    int: The sum of x and y.
    """
    return x + y

# Example usage of the function
result = my_function(3, 4)
print("The result is:", result)
\end{lstlisting}

%%%%%%%FIN-ANNEXES%%%%%%%%%%
\end{document}
