\documentclass[a4paper,11pt]{article}
\usepackage[utf8]{inputenc}
\usepackage{amsmath}
\usepackage{amsfonts}
\usepackage{amssymb}
\usepackage{hyperref}
\usepackage{enumitem}
\usepackage{geometry}
\usepackage{titlesec}

% Page geometry
\geometry{top=1in, bottom=1in, left=1in, right=1in}

% Title formatting
\titleformat{\section}{\large\bfseries}{\thesection}{1em}{}

% Hyperlink setup
\hypersetup{
    colorlinks=true,
    linkcolor=blue,
    urlcolor=blue,
}

% Remove page numbers
\pagenumbering{gobble}

% Reduce space between items
\setlist{noitemsep}

% Begin document
\begin{document}

% Title
\begin{center}
    \LARGE \textbf{Identifier des noeuds IoT en espionant leur signal radio} \\
    \vspace{0.2cm}
    \large Tulippe Hecq Arnaud \\
    \normalsize Département d'Informatique \\
    \vspace{0.5cm}
\end{center}

% Abstract

\noindent L'avènement de l'Internet of Things a lancé une nouvelle ère d'appareils connectés, ouvrant de nouvelles possibilités de partage de l'information, d'automatisation et de protection. L'expansion de l'IoT soulève une nouvelle problèmatique de sécurité. Entre autres, l'identification des noeuds au sein des réseaux est essentielle. Il a été découvert que des noeuds fabriqués avec les mêmes microprocesseurs et modèles d'émetteurs-récepteurs radio peuvent présenter de subtiles particularités dans les caractéristiques de leurs signaux. Cette variabilité intrinsèque de la transmission des signaux radio peut être exploitée pour distinguer les noeuds d’un réseau. En écoutant leurs signaux radio et en analysant leurs signatures distinctes, il devient possible de les identifier.

\vspace{0.2cm}
% Introduction
\noindent L'objectif du travail est d'identifier des noeuds de l'Internet of Things utilisant la technologie LoRa, en se basant uniquement sur les caractéristiques spécifiques de leur signal radio, différenciées par une dispersion sensible de la performance des composants électroniques intégrés dans chaque émetteur. LoRa est une technologie de communication sans fil qui permet de transmettre des données sur de longues distances avec une faible consommation d'énergie. LoRa fonctionne avec un protocole de type Low Power Wide Are Network appelé LoRaWAN. Ce travail est structuré en trois parties.

\vspace{0.2cm}

\noindent Le premier chapitre développe les aspets du traitement du signal nécessaires à la comprehénsion des expérimentations menées durant le mémoire. Le concept de signal radio, le principe de modulation, la gestion du bruit et la Transformée de Fourier sont détaillés dans ce chapitre. Cette partie comprend également un description de la couche physique de la technologie LoRa, ainsi qu'une présentation de son protocoal LPWAN LoRaWAN.

\vspace{0.2cm}

\noindent Le deuxième chapitre se concentre sur les expérimentations effectuées. D'abord tout le matériel est introduit. Les différentes radio logicielles (SDR) DVB-T, R820T2 et HackRF sont détaillés avec leur schéma blocs respectifs. Les modules d'émission LoRa de type RN2483, Arduino et Pycom LoPy sont présentés avec leur caractéristiques respectives. Ensuite, les logiciells d'analyse tel que Universal Radio Hacker (URH) ou GQRX sont indroduits avec leur fonctionnement. Le chapitre se termine avec l'analyse des signaux LoRa généré par les différents modules d'émission via les logiciels et via Python. 

\vspace{0.2cm}

\noindent Le dernier chapitre présente la méthode utilisée pour l'indentification des noeuds. Les diagrammes de constellations différenciées (DCTF) permettent de mettre en évidence l'unique propriété physique appelée \textbf{signature} pour chaque noeud. L'article de par Yu Jiang, Linning Peng, Aiqun Hu, Sheng Wang, Yi Huang et Lu Zhang \textit{Lora Devices Identification Based on Differential Constellation Trace Figure} affirme que la position géographique de la signature permet d'indentidication des noeuds. Cette méthode est appliquée sur le matériel présenté au deuxième chapitre.

\vspace{0.2cm}

\noindent Les résultats du mémoire ne s'alignent pas avec l'article \textit{Lora Devices Identification Based on Differential Constellation Trace Figure}. Cependant, une approche alternative s'est révélée durant les expérimentations. Cette piste, se concentre toujours sur la signature comme propriétés discriminante pour des noeuds LoRa, mais en s'intéressant à sa forme géométrique au sein de la DCTF plutot qu'à sa position géographique.

\end{document}