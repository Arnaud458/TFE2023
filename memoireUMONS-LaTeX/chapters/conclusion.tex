\chapter*{Conclusion}
\addcontentsline{toc}{chapter}{Conclusion}
\renewcommand{\leftmark}{CONCLUSION}

L'objectif du travail était de pouvoir identifier des noeuds de l'Internet Of Things utilisant la technologie LoRa en se basant uniquement sur leur signal radio.

\vspace{0.1cm}

Cette étude, le radio frequency finguer printing indentification a été réalisé par Yu Jiang, Linning Peng, Aiqun Hu, Sheng Wang, Yi Huang et Lu Zhang dans \cite{loraDCTF} via l'utilisation de diagrammes de constellations. Ils ont montré qu'il était possible en analysant les coordonées géographiques d'une signature radio dans une DCTF de pouvoir retrouver l'appareil LoRa émetteur.

\vspace{0.1cm}

Afin de reproduire cette étude, il a fallu dans un premier temps apprendre le fonctionnement de la technologie LoRa et  de la modulation qu'emploient les émetteur LoRa. Ensuite la compréhension du fonctionnement et de l'utilisation d'une radio logicielle avant de finalement commencer à captuer des signaux.

\vspace{0.1cm}

L'analyse des signaux a été possible grace à des logicielle facile d'utilisaton et flexible comme GQRX et URH, avant d'avoir recour à des librairies python pour pouvoir identifier plus en profondeur les caractéristiques de ces signaux.

\vspace{0.1cm}

Finalement, la reproduction de l'étude de Yu Jiang, Linning Peng, Aiqun Hu, Sheng Wang, Yi Huang et Lu Zhang aura révélé des résultats qui contredisent les affirmations de l'article \cite{loraDCTF}. Ces résultats montrent cependant qu'une identification basée sur le RFFI est bien possible, mais pas selon les coordonnée géographiques de la signature, mais plutot sur sa forme géométrique.

\vspace{0.1cm}

Ainsi, une nouvelle piste dans l'indentification de noeuds LoRa basé sur la méthode des diagrammes de constellation pourrait se concentrer sur la reconaissance d'image afin de retrouver pour des paramètres d'émission fixe une image correspodant à la signature du signal. L'utlisation des réseaux de neurones convolutif est déja présente dans le RFFI. Son ajout à la méthode des DTCF pourrait permettre de facilement reconnaitre la signature.