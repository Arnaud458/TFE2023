\chapter*{Conclusion}
\addcontentsline{toc}{chapter}{Conclusion}
\renewcommand{\leftmark}{CONCLUSION}

L'objectif du travail était de pouvoir identifier des noeuds de l'Internet of Things utilisant la technologie \ac{LoRa} en se basant uniquement sur les caractéristiques spécifiques de leur signal radio, différenciées par une dispersion sensible de la performance des composants électroniques intégrés dans chaque émetteur.
\vspace{0.1cm}

Cette étude, le Radio Frequency Fingerprinting Identification, a notamment été réalisée par Yu Jiang, Linning Peng, Aiqun Hu, Sheng Wang, Yi Huang et Lu Zhang dans leur article \cite{loraDCTF} via l'utilisation de diagrammes de constellation. Ils ont montré qu'il était possible, en analysant les coordonnées géographiques d'une signature radio dans une \ac{DCTF}, de pouvoir retrouver l'appareil \ac{LoRa} émetteur. Ces auteurs se sont concentrés sur la fréquence de transmission unique des émetteurs \ac{LoRa}, de leur déviation spécifique vis-à-vis de la fréquence théorique, comme signature d'identification. Les diagrammes de constellation ont ainsi été générés à partir des composantes \ac{I/Q} des signaux modulés transmis par ces émetteurs selon leurs caractéristiques fréquentielles différenciées.

\vspace{0.1cm}

Afin de reproduire cette étude, il a fallu dans un premier temps apprendre en détails le fonctionnement de la technologie \ac{LoRa} et  de la modulation qu'emploient les émetteurs \ac{LoRa}. Ensuite, il a fallu maitriser le fonctionnement et l'utilisation d'une radio logicielle avant de finalement commencer à capturer des signaux radio.

\vspace{0.1cm}

L'analyse des signaux a été possible grace à des logiciels \textit{open source} et flexibles comme GQRX et \ac{URH}, avant d'avoir recours à des librairies Python pour pouvoir étudier plus en profondeur les caractéristiques de ces signaux.

\vspace{0.1cm}

Finalement, la reproduction de l'étude aura révélé des résultats qui ne s'ali\-gnent pas avec les affirmations de l'article \cite{loraDCTF}. Ceux-ci montrent cependant qu'une identification basée sur le \ac{RFFI} est bien possible, non pas selon les coordonnées géographiques de la signature, mais plutôt selon sa forme géométrique.

\vspace{0.1cm}

Ainsi, une nouvelle piste dans l'identification de noeuds LoRa basée sur la méthode des diagrammes de constellation pourrait se concentrer sur la reconaissance d'images afin de retrouver, pour des paramètres d'émission fixes, une image correspondant à la signature du signal. L'utilisation des réseaux de neurones convolutifs est déja présente dans le \ac{RFFI}. Son ajout à la méthode des \ac{DCTF} pourrait permettre de facilement reconnaitre la signature.