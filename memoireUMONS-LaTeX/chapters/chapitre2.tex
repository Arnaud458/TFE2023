\chapter{Travaux similaires et autres contributions}

Ce chapitre a pour but d'établir un état de l'art dans le domaine de l'internet of things. Comme l'iot est très vaste, seulement certains aspects seront présentés. Tout d'abord, il est important de comprendre que la sécurité dans ce domaine a évoluée autan bien par ces méthodes que par son importance, ainsi un historique est présenté dans la section\ref{histoire} de ce chapitre. Ensuite, la seconde section se concentre uniquement sur la technologie Lora. Cette section donne une approche analytique des signaux. Finalement, la dernière partie de ce chapitre se consacre aux approches existantes pour l'identification d'appareils utilisant Lora. Cette section présentera les travaux qui sont au centre des expérimentations de ce travail.


\section{Identification d'appareils dans l'iot}

Avante de s'intéresser à l'identification d'appareils Lora, il est nécessaire d'étendre ce concept à l'iot. L'internet of things a connu une forte évolution depuis les début des années 2000, avec des priorités dans son évolution qui ont également changées au fil du temps.

\subsection{historique et évolution des préoccupation de sécurité dans l'iot}\label{histoire}

Afin de bien comprendre les enjeux des différentes époques, voici une petit historique des deux décénnies précedantes. Les technologie mais également le concept même d'internet of thing ont évolué.

\vspace{0.1cm}

Les premières années de l'iot (\textit{début des années 2000}).

Bien que le concept d'appareils connectés remonte aux années 70, l'avènement de l'internet of thing arrive en fin de milénaire. Ce concept est associé à la technologie RFID \textit{Radio Frequency Identification}\cite{RFID}, qui permet d'utiliser les ondes radios afin d'identifier des objets ou des personnes. Le but initial était de rendre tout objet dans le monde identifiable par un code EPC ou \textit{Electronic Product Code}\cite{EPC}, un peu comme le code barre. Durant ses années, plusieurs entreprises lancent leur première appareils connectés. Tout cet enthousiame pour la connectivité met au second plan les questions de sécurité. Ainsi la première partie du développemnt de l'iot se concentre surtout sur la qualité de communication entre les objets plutot que sur leur sécurité.

\vspace{0.1cm}

Les premières préoccupations sécuritaires (\textit{fin des années 2000, début des années 2010})

Vers la fin des années 2000, l'augmentation du nombre d'appareils est si grande qu'elle a atteint tous les domaines de la société. Certains domaines étant plus critique que d'autre d'un point de vue sécurité (l'énergie, les transports, la santé, etc),l'intégrité des données, la confidentialité et les accès réseaux deviennent le centre de l'attetion. 
Le concept de certificats x.509, initalement développé pour le world wide web avant les années 2000, a un regain d'attention dans cette période. Plus largement, la structure de la technologie PKI (\textit{Public Key infrastructure}, qui utilise les certificats x.509)\cite{PKI} a été adaptée pour s'intégrer aux problématiques de l'embarqué. Un certificat est un document digital permettent de vérifier d'identité d'une entité, comme d'un appareil, un utilisteur ou une organisation. Il se base sur la liaison d'une clé public à l'entité établie par une \textit{Certificate Authority (CA)}. La CA agit en temps de tier de confiance et assure la légitimité de l'information grâce au certificat. Ainsi, les trois axes principaux de la sécurité dans l'IoT émergent : l'autentification, l'intégrité des données et la confidentialité.

\vspace{0.1cm}

L'avènement du \textit{Edge Computing}(à partir des années 2010).

Le nombre d'appareils connecté  dépasse le nombre d'êtres humains, forçant une transition vers l'\textit{ipv6} tant le nombre d'appareils est évelé et continue d'augmenter. L'information a pris de la valeur, et de l'ampleur. Ainsi, viens se gréffer de nouveau enjeux économiques en plus des enjeux sécuritaires. La quantité de données générées nécessite de revoir le stockage de l'information. C'est ainsi que va apparaitre le Edge computing\cite{edge}, qui est un réponse directe aux besoin des architectures de gérer autan de données en périphérie de réseau. Le concept du edge computing vise à effectuer des calculs et des analyses des données directement sur les appareils connectés, plutôt que de les envoyer vers un centre de données centralisé. Cela réduit la latence, améliore l'efficacité du réseau et permet des analyses en temps réel. Le premier malware spécialemnt centré sur l'iot fait son apparition. Mirai\cite{Mirai} exploite une faille lui permettant de récupérer le mots de passe d'appareils afin de s'en servir pour lancer des attacks \textit{DDoS (distributed denial of service)} à grande échelle. En quelques années l'IoT est passé d'un gadget d'entreprise à un vériable enjeu économique et sécuritaire, centré autour de l'information. Les seules perspectives de législation concernant la sécurité de l'internet of thing n'apparaitront de tradivement en fin de décennies avec La loi européenne sur le \textit{Réglément générale sur la protection des données}(source : loi \href{https://commission.europa.eu/law/law-topic/data-protection/data-protection-eu_en}{RGPD}. cette loi ne couvre pas la sécurité des appareils mais plutot l'utilisation des données sur internet en général.

\vspace{0.1cm}

L'apparition de la \textit{Blockchain}(fin année 2010).

Le stockage et la transmission de données, l'authentification d'appareils ou encore la confidentialité sont au centre des préoccupations. Initialement utilisée dans les cryptomonnaie, la technologie Blockchain est un mécanisme de base de données qui permet un partage transparent des informations au sein d'un réseau. Une base de données Blockchain stocke les données dans des blocs qui sont reliés entre eux dans une chaîne. Les données sont chronologiquement cohérentes, car il n'est pas possible de supprimer ou modifier la chaîne sans le consensus du réseau. Par conséquent,la technologie Blockchain peur servir de livre inaltérable ou immuable pour le suivi des ordres, des paiements, des comptes et d'autres transactions. Le système dispose de mécanismes intégrés qui empêchent les entrées de transactions non autorisées et créent une cohérence dans la vue partagée de ces transactions. L'implémentation de la blockchain pour l'iot confère les avantages suivant\cite{block} : 
\begin{itemize}
\item l'immuabilité. la blockchain permet de créer un enregistrement immuable de toutes les interactions et communications des appareils. Cet enregistrement peut être utilisé pour détecter et empêcher l'accès non autorisé ou la modification des appareils ou des données dans l'IoT.
\item La décentralisation. il est possible de créer un système décentralisé pour l’authentification et la communication des appareils. Chaque appareil IoT se connecte au réseau blockchain et se voir attribuer une identité numérique unique, qui est vérifiée grâce à l'utilisation de signatures numériques ou de contrats intelligents. Cela élimine le besoin d’une autorité centrale pour authentifier les appareils et annule ainsi les risques de \textit{single point of failure}.
\item La confidentialité.La technologie Blockchain peut sécuriser la communication entre les appareils IoT grâce à l'utilisation de la cryptographie à clé publique ou asymétrique. Cela permet l’échange sécurisé d’informations entre appareils sans avoir recours à des intermédiaires.
\end{itemize}

\vspace{0.1cm}

Le \textit{Zero Trust Model}\cite{zero2}(début des années 2020).

les menaces de sécurité sont de plus en plus sophistiquées. Comment faire encore confiance aux infrastructures qui doivent gérer autan d'appareils ? La réponse est de ne plus leur faire confiance. Le zero Trust model est donc un modèle basé sur l'absence totale de confiance et une vérification constante, que la demande d'accès provienne de l'intérieur ou de l'extérieur du réseau. Dans un modèle de sécurité classique, une fois qu'un utilisateur ou un appareil accède au réseau interne, on lui fait souvent implicitement confiance, ce qui lui permet une grande liberté d'actions au sein du réseau. Le Zero Trust model suppose cependant que des menaces peuvent exister à la fois à l’intérieur et à l’extérieur du périmètre du réseau et nécessite donc une vérification continue de la confiance. Un modèle qui s'applique sur ce principe devrait contenir les éléments suivants\cite{zero1} :
\begin{itemize}
\item La vérification d'identité. Les utilisateur et les appareils doivent subir une authentification avant d'accèder à n'immporte quel services ou ressources du réseau.
\item Le \textit{Least privilege acces}. les permissions sont accordées de manières limités selon le besoin de l'utilisateur ou de l'appareil.
\item La micro segmentation. Diviser le réseau en segments pour limiter son accès par les appareils.
\item La surveillance en continue. L'analyse du traffic, du comportement et des activités des appareils.
\item le chiffrage des données.
\end{itemize}

\vspace{0.1cm}

Le \textit{quantum computing}(dans les prochaines années...).

Avec la future arrivée des ordniateurs quantiques, les méchanismes de chiffrage basés sur la comlexité mathématique comme RSA ou ECC sont voués à disparaitre\cite{quantumcrypto}. La puissance de calcul des ordinateurs quantique est déja considérée comme une véritable menace pour la sécurité informatique. Fort heureusement, c'est également un nouveau champ de possibilité qui s'ouvre pour la sécurité, avec le développement du \textit{post quantum cryptography}. Un premier protocol résistant au menances quantiques, \textit{Quantum Key Distribution} permet d'établir des canaux de communications entre différents appareils dand l'iot. Ce protol n'est pas encore en service dans l'iot, mais les premiers est réalisé en laboratoire sont très prometteurs\cite{qinternet}.

\subsection{Approches d'identification dans l'iot}

Comme tout appareil au sein d'un réseau, un appareil connecté dans l'iot possède de base différents moyens d'authentification. Son adresse MAC, son adresse IP, des informations relatives à sa manufacturation comme un numéro de série par example. De manière plus spécifiques, les noeuds au sein d'un réseau LoraWAN possède un DevEUI, un numéro unique accordé par le réseau à l'appareil. Cependant ses informations peuvent être compromises si appareils sont victimes de \textit{devices spoofing}, c'est à dire qu'un appareil malveillant usurpe l'indentité de sa cible, afin d'accéder au sein du réseau. D'autres attaques comme \textit{Man in The Middle} ou \textit{Replay attacks} peuvent également compromettre l'identité d'un appareil si on se base uniquement sur ses identifiant classiques\cite{attack}. Il faut donc pouvoir identifier les appareils mais sans se fier à leur informations. Il éxiste diverse méthodes basée sur différentes approches pour pouvoir indentifier un appareil. 

La première approche possible est de fier non pas à l'appareil directement ni aux données qu'il reçoit car celles ci pourraient également être compromises, mais à la routine sur sa communication. Cette approche, appelée \textit{Traffic related pattern} où s'intéresse au comportement d'un appareil au sein d'un réseau. L'article publié par H. Kawai, S. Ata et N. Nakamura \cite{pattern} propose notament d'analyser via machine learning le \textit{traffic pattern} c'est à dire le comportement du traffic.

Une autre approche s'intéresse à la position géographique d'un appareil. En effet il est possible de mesure la qualité de la réception d'un signal, le \textit{Received Signal strengh Indicator} ou RSSI. Un étude a été menée sur des appareils lora par M. Anjum, MA. Khan et SA. Hassan\cite{rssi}. L'article utilise le \textit{Path Loss} pour estimer le RSSI. Via machine learning ils sont capables de recréer un système de positionement permettant l'identification d'appareil Lora.

La dernière approche se concentre sur l'analyse des charactéristiques uniques des fréquences radios. Plutot que de s'intéresser au routine entre les communication ou la distance d'où elles ont lieu, le \textit{Radio Frequency Finguerprinting} se concentre sur les propriétés physiques des signaux. Différentes techniques sont montrées par N. Soltanieh, Y. Norouzi et Y. Yang\cite{rffi1}.

\textcolor{red}{FIN DE LA REDACTION}

\section{analyse de la technologie lora}

\subsection{analyse de la couche physique lora}

des travaux ont été réalisé pour permettre l'analyse de la couche physique de lora (article)
fait en reverse engeneering. 

\subsection{analyse de la modulation CSS de Lora}

papier belgique sur la modulation Css

\section{identification de device lora}

radio frequency finguerprinting identification (RFFI). trouver des charactéristiques hardware pour identifier des devices.

\subsection{RFFI avec DCTFs}
article sur les méthode de constellations traces

\subsection{RFFI avec spectrogrammes}
article sur les spectogrammes.