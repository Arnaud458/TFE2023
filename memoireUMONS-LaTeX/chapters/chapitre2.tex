\chapter{Travaux similaires et autres contributions}

trois partie,une partie générique comprenant certaines approche pour identifier des devices dans l'iot en général. Une seconde partie reprenant des approce d'analse de signaux lora. Finalement une section joigant les deux première partie pour atteindre l'objectif du travail, l'identification de devices spécifique à lora via l'analyse de leur signaux radio.


\section{identification de device dans l'iot}

avant de s'intéresser à l'identification de device lora, besoin de regarder plus largement dans l'iot et de voir si l'identification de devices est étendue à l'iot en général (oui), et comment l'indentification est acomplie.

piste : RSSI, TCP finguerprinting, taffic related patterns (behaviour, power consumption, packet timing) 

\section{analyse de signaux lora}

differente travaux sur comment interpréter un signal. des analyses des différent composantes du signal 

\section{identification de device lora}

radio frequency finguerprinting identification (RFFI). trouver des charactéristique hardware pour identifier des devices.
article sur les méthode de constellations traces
article sur les spectogrammes.