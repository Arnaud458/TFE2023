\section{Annexe A :Code module Arduino}\label{codearduino}

\begin{lstlisting}[style=cppstyle, caption={Module Arduino}, label={lst:cpp}]

#define LED_BUILTIN 3

#define SX1276_NSS   10
#define SX1276_RESET 9
#define SX1276_DIO0  2

// the setup function runs once when you press reset or power the board
void setup() {
  // initialize digital pin LED_BUILTIN as an output.
  pinMode(LED_BUILTIN, OUTPUT);
  Serial.begin(57600);
  while (!Serial);

  LoRa.setPins(10, 9, 2);

  if (!LoRa.begin(868.1E6)) {
    Serial.println("Lora init failed. Check your connections.");
    while (true);
  }

  Serial.println("LoRa setup");
  //LoRa.setSyncWord(0x12); // private
  LoRa.setSyncWord(0x34); // LoRaWAN
  Serial.println("  Sync Word = 0x34 (LoRaWAN)");
  LoRa.setSpreadingFactor(12);
  Serial.println("  SF = SF12");
  LoRa.setSignalBandwidth(7.8E3);
  Serial.println("  BW = 7.8kHz");
  LoRa.dumpRegisters(Serial);

}

int counter = 0;

// the loop function runs over and over again forever
void loop() {

  digitalWrite(LED_BUILTIN, HIGH);   // turn the LED on (HIGH is the voltage level)
  delay(1000);                       // wait for a second
  digitalWrite(LED_BUILTIN, LOW);    // turn the LED off by making the voltage LOW
  delay(1000);                       // wait for a second
  Serial.println("Sending frame...");

  unsigned long t0 = micros();
  LoRa.beginPacket(0); // 0 = explicit header, 1 = implicit header
  LoRa.print("B");
  //LoRa.print(counter);
  LoRa.endPacket(0); // 0 = wait for end of packet transmission

  // Time on Air as calculated per https://loratools.nl/#/airtime
  // should be around 13 seconds, but measurements give around 10 seconds (why ?)
  Serial.print("ToA ~ ");
  Serial.println(micros()-t0);

  //LoRa.idle(); // Warning, further transmission do not work; need to reset module ?
  //LoRa.setSyncWord(0xFF); /* LoRaWAN */


  counter++;
}
\end{lstlisting}

\newpage

\section{Annexe B :Code module RN2483}\label{codern}

\begin{lstlisting}[style=pythonstyle, caption={Configuration module RN2483}, label={lst:python}]
import serial
import time
from utils import read_file, write_file

SERIAL_PORT = '/dev/ttyUSB0'
BAUD_RATE = 57600
SPREADING_FACTOR = 8

config_commands = [
    "sys get ver\r\n",
    "radio set mod lora\r\n",
    "radio set freq 868000000\r\n",
    "radio set pwr 14\r\n",
    f"radio set sf sf{SPREADING_FACTOR}\r\n",
    "radio set cr 4/8\r\n",
    "radio set bw 125\r\n"
]

MESSAGE_COMMAND = "radio tx 1509ACf\r\n"


def config(ser: serial.Serial):
    # Send commands and read responses
    for command in config_commands:
       write_command(ser, command)


def send_signal(ser: serial.Serial):
    write_command(ser, MESSAGE_COMMAND)


def write_command(ser: serial.Serial, command: str):
    ser.write(command.encode())
    response = ser.read(100)
    print("Command:", command.strip())
    print("Response:", response.decode().strip())
    print("------------")
\end{lstlisting}

\newpage

\section{Annexe C :Implémentation de la configuration RTL SDR via PyRTLSDR}\label{antenne}

\begin{lstlisting}[style=pythonstyle, caption={Configuration RTL SDR}, label={lst:python}]
import math
import time
import numpy as np
from rtlsdr import RtlSdr
import hackrf
from utils import cut_preamble, save_signal_old, read_file, write_file
from find_centers import SAMPLES_FOLDER


SAMPLE_RATE = 2_000_000


def capture_signal_rtlsdr():
    # Configure RTL-SDR parameters
    sdr = RtlSdr()
    sdr.sample_rate = SAMPLE_RATE
    #sdr.bandwitdh = 250000
    sdr.center_freq = 867937500
    sdr.gain = 5

    # Start signal capture
    capture_duration = 2.5  # in seconds

    print(f"Capturing signal for {capture_duration} seconds...")
    nb_samples = math.ceil(capture_duration * sdr.sample_rate /16384)*16384
    #samples = sdr.read_samples(5046272)
    samples = sdr.read_samples(nb_samples)

    sdr.close()
    samples = np.array(samples, dtype=np.complex64)
    return samples
\end{lstlisting}

\newpage

\section{Annexe D :Implémentation de l'automatisation des captures de signaux}\label{codeauto}

\begin{lstlisting}[style=pythonstyle, caption={Emetteur}, label={lst:python}]
if name == 'main':
    SER = serial.Serial(SERIAL_PORT, BAUD_RATE, timeout=1)
    for i in range(5):
        time.sleep(1)
        config(SER)
        write_file('tmp.txt','0')
        time.sleep(1.5)
        send_signal(SER)
        while read_file('tmp.txt')[-1] != '1':
            print('emmeteur is waiting')
            time.sleep(1)
    #config(SER)
    #send_signal(SER)
    SER.close()
\end{lstlisting}

\begin{lstlisting}[style=pythonstyle, caption={Récepteur}, label={lst:python}]

if name == "main":
    PREAMBLEDURATION = 0.0245
    for i in range(2):
        while read_file('tmp.txt')[-1] != '0':
            print('recepteur is waiting')
            time.sleep(0.2)

        SIGNAL = capture_signal_rtlsdr()
        #SIGNAL = capture_signal_hackrf()
        PREAMBLE = cut_preamble(SIGNAL, 0.03, int(PREAMBLE_DURATION*SAMPLE_RATE))

        if len(PREAMBLE) > 0:
            save_signal_old(PREAMBLE,f'{SAMPLES_FOLDER}sample{i+1}', np.complex64)
            print(f'signal {i+1} saved')
        else:
            print(f"no signal found for {i+1}")

        write_file('tmp.txt', '1')
\end{lstlisting}

\begin{lstlisting}[style=pythonstyle, caption={Preprocessing}, label={lst:python}]

import numpy as np


def read_file(filename: str):
    with open(filename, 'r',encoding='utf8') as reader:
        return reader.readlines()

def write_file(filename: str, content: str):
    with open(filename, 'a',encoding='utf8') as writer:
        writer.write('\n' +content)
        
        def save_signal(signal: np.ndarray, filepath: str, dtype) -> None:
    signal_to_save = np.array(signal, dtype=dtype)
    print(f"Saving captured signal to {filepath}.npz")
    np.savez_compressed(filepath, signal_to_save)


def load_signal(filepath: str, dtype) -> np.ndarray:
    if filepath.endswith('.npz'):
        return np.load(filepath)['arr_0']
    return np.fromfile(filepath, dtype=dtype)
    
def cut_preamble(signal, treshold, preamble_size):
    start_index = 0
    consecutive_start = 0
    while start_index < len(signal) and consecutive_start <3:
        if abs(np.real(signal[start_index])) > treshold or abs(np.imag(signal[start_index])) > treshold:
            consecutive_start += 1
        else :
            consecutive_start = 0
        start_index += 1

    if start_index + preamble_size > len(signal) -1:
        return []

    return signal[start_index:start_index+preamble_size]

def rms_normalize(samples: np.ndarray) -> np.ndarray:
    rms_values = np.sqrt(np.mean(np.abs(samples)**2, axis=0))  # Compute RMS values
    normalized_samples = samples / rms_values  # Normalize samples
    return normalized_samples
\end{lstlisting}