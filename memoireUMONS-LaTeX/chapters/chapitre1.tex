\renewcommand{\leftmark}{NOMINATION DES TECHNOLOGIES}

\chapter{Rappels et nomination des technologies}

\section{Signal radio}

Un signal est une variation dans l'espace ou dans le temps d'une quantité physique contenant de l'information. Un signal peut être continu ou discret, on le nomme alors respectivement analogique ou numérique. Le type de signal dépent notamment de l'information qu'il contient. Un signal analogique peut contenir par exemple du son, là où un signal numérique contient généralement un nombre fini de valeur (par exemple des 0 et 1).
Les deux catégories ne sont pas incompatible car il est souvent nécessaire en télécommication de pouvoir passer de l'un à l'autre.\cite{test}

\vspace{0.1cm}

L'utilisation de signaux radio en télécomunication confère de nombreux avantages, comme la portée, la vitesse de transmission , la résistance aux interférence ou encore le coût de propagation. Tout ces avantages sont possibles car un signal peut être modulé. La modulation est une technique permettant de modifier les propriétés du signal lui permettant de transporter de l'information.

\newpage

En télécommunication, les signaux sont associés aux ondes radios, ainsi appelé $radio$ $signal$ ou signal radio. Voici les principaux attributs d'un signal radio: 

\vspace{0.1cm}

\begin{itemize}

\item la fréquence, mesurée en Hertz ($Hz$). Elle détermine le nombre de cycle qu'accomplie le signal par seconde.
\item La largeur de spectre, elle dépent de la fréquence car c'est l'écart entre la plus haute et la plus basse fréquence du signal. Une plus grande largeur permet de transmettre plus d'informations.
\item L'amplitude. Selon le type de signal l'attribut possède différentes fonctions. Dans le cas d'un signal analogique l'amplitude est l'une des caractéristiques principales d'identification du signal mesurant l'ampleur du signal. Dans un signal numérique l'amplitude sert plutot de marge entre les différentes états du signal. 
\item la puissance, mesurée en Décibel ($dB$). C'est la force du signal, un attribut important pour la réception du signal notamment.
\item le $signal$ $to$ $noise$ $ratio$ ou $SNR$. Cet attribut mesure la qualité du du signal. une valeur élevée indique que le pourcentage de bruit est faible.
\item le $bit$ $rate$, ou le taux de transmission mesure la quantité de donnée transmise en bit par seconde. Cet attribut est exclusif aux signaux numériques. On parle de $Baud$ $rate$ pour les signaux analogiques. Ce n'est pas excatement l'équivalent du bit rate car c'est le nombre de symboles modifiés par seconde, et un symbole peut contenir plusieurs bit pour un signal numérique.

\end{itemize}

\section{Traitement du signal}

\subsubsection{Modulation}

La réception d'un signal nécessite des antennes dont les dimensions dépendent de la longueur d'onde du signal. Un signal à haute fréquence a l'avantage d'être facilement transmissible sur une grand portée. Cependant les signaux originaux (appelés $baseband$ $signals$) sont en basse fréquence. La modulation d'un signal permet de transformer le signal en haute fréquence, devenant ainsi le signal modulé.

\vspace{0.1cm}

Parmis ces différents attributs, certains sont utilisés pour effectuer une modulation. Les deux modulations les plus utilisés sont basées sur les attributs de la fréquence et de l'amplitude. La modulation en fréquence (ou $FM$ pour $frequency$ $modulation$) consiste à encoder l'information en faisant varier la fréquence en maitenant l'amplitude constante. La modulation en amplitude ($AM$) est le procédé inverse, c'est à dire encoder l'information en faisant varier l'amplitude tout en gardant la fréquence constante. 

\vspace{0.1cm}

La modulation en amplitude est plus ancienne et est encore utilisée dans beaucoup de systèmes. Cette technique possède moins de contrainte et est notamment plus simple à implémenter. Elle requiert le signal modulant et un signal haute fréquence appelé $carrier$ $signal$.

\vspace{0.1cm}

Soient un signal modulant $u(t)$ et un sigal porteur (ou $carrier$ $signal$) $v(t)$, la modulation en amplitude s'effectue en multipliant les deux signaux pour obtenir le signal modulé 

\vspace{0.1cm}

$s(t)$ = $u(t)$ . $v(t)$

\begin{figure}[h]
\centering

\includegraphics[scale=1]{images/AM_mod.PNG}
\caption{Exemple de modulation en amplitude}\label{term1}
\end{figure}


La modulation en fréquence permet d'obtenir des transmissions de meilleures qualités plus résistantes à leur environement tout en gardent une puissance d'émission constante. 

\vspace{0.1cm}

Soient un signal modulant $u(t)$ et un signal porteur sinuosidal

\vspace{0.1cm}

$v_{p}(t)$ = $A_{p}cos(2\pi f_{p}t)$ où

$f_{p}$ est la fréquence de la porteuse,

$A_{p}$ est l'amplitude de la porteuse,

alors le signal modulé $s(t)$ = $A_{p}cos(2\pi \int_{0}^{t}f(\tau)d\tau)$ où $f$ est la fréquence instantanée. Elle s'exprime en fonction de la dérivation de fréquence $f_{\Delta}$, c'est à dire la dérivation maximale par rapport à la fréquence de la porteuse.

\newpage

$f_{p}$. $f(t)$ = $f_{p}$ + $f_{\Delta} u(t)$

\begin{figure}[h]
\centering

\includegraphics[scale=1]{images/FM_mod.PNG}
\caption{Exemple de modulation en fréquence}\label{term2}
\end{figure}



\subsubsection{Gestion du bruit}



L'un des attributs cités concerne le bruit. Un signal est toujours affecté de petites fluctuations plus ou moins importantes, et dont les origines peuvent être diverses. Ces perturbations, appelée bruit ou $noise$ en télécommunication se définissent par l'altération non souhaitée de l'intégrité d'un signal. Le bruit peut prendre différentes formes, des perturbations essentiellement impulsionnelles engendrées par des commutations de courants ou alors du bruit de fond généré dans les câbles et les composants électroniques en raison
des mécanismes statistiques de la conduction électrique. Il est possible de réduire voir élimier l'influence des perturbations impulsionelles. En revanche, le bruit de fond est lui irreductible. Tout signal sans bruit n'existe pas, même à l'émission. Il est cependant possible que le bruit devienne invisible si son niveau est très faible. L'attribut SNR est donc un critère de la qualité du signal.

\newpage

\subsubsection{transformée de Fourier}

Pour effectuer une analyse de signal, sa représentation est capitale. Les Figure 1 et 2 représentent des signaux en fonction du temps écoulé. Il est possible de représenter des signaux selon une autre composante, la fréquence par exemple.

\vspace{0.1cm}

La transformée de Fourier est un outil fondamental utilisé pour analyser et décomposer des signaux complexes en composantes fréquentielles. En transformant un signal dans le domaine temporel en sa représentation dans le domaine fréquentiel, la transformée de Fourier révèle les différentes composantes fréquentielles présentes dans le signal. En fonction du type de signal, la transformée de Fourier est adaptée.

\vspace{0.1cm}

Pour les signaux continus, la $CFT$ (Transformée de Fourier continue) convertit une fonction du temps en fonction de la fréquence en intégrant le signal par rapport aux sinusoïdes de toutes les fréquences possibles. Cette transformation fournit les informations d'amplitude et de phase pour chaque composante de fréquence présente dans le signal.

\vspace{0.1cm}

Pour les signaux discrets et échantillonnés, la $DFT$ (Transformée de Fourier discrète) calcule un ensemble fini de composantes de fréquence. Il est calculé à l’aide d’un nombre fini d’échantillons, ce qui donne des composantes de fréquence discrètes. Il existe un méthode simplifiée pour les signaux discrets appelé $FFT$ (Fast Fourier Transform). Il s'agit d'un moyen plus rapide de calculer la transformée de Fourier, en particulier pour les signaux numériques comportant un grand nombre de points de données. L'avantage principal de cet algorithme permet de réduire le temps de calcul en divisant la DFT en sous problèmes. La FFT est une méthode très utilisée pour l'analyse de signaux.

\newpage

\section{LoRa}

$LoRa$ (Long Range) est une technologie de communication sans fil qui permet de transmettre des données sur de longues distances avec une faible consommation d'énergie. Elle a été développée par la société française Cycleo et est maintenant gérée par la fondation LoRa Alliance, qui regroupe plusieurs entreprises et organisations du monde entier.

\vspace{0.1cm}

LoRa est principalement utilisée dans l'$Iot$. Elle se distingue par sa portée étendue, qui peut atteindre plusieurs kilomètres en milieu urbain et plusieurs dizaines de kilomètres en milieu rural, ainsi que par sa faible consommation d'énergie, qui permet de prolonger la durée de vie des appareils connectés. Une longue portée avec un puissance limitée induit une plus faible bande passante que les autres technologies sans fil (le Wifi, la 4G, Bluetooth etc).

\vspace{0.1cm}

LoRa utilise une bande de fréquences qui varie selon les régions du monde où LoRa est déployée :

\vspace{0.1cm}

\begin{itemize}
\item en Europe, la bande de fréquences autorisée est comprise entre 863 et 870 MHz,
\item aux États-Unis, elle se situe entre 902 et 928 MHz,
\item en Chine, la fréquence autorisée varie entre 779 et 787 MHz,
\item les régions restantes ont elles aussi une fourchette unique.
\end{itemize}

\vspace{0.1cm}

La technologie LoRa utilise la modulation en fréquence chirp spread spectrum ($CSS$ $modulation$). La modulation CSS utilise un signal chirp, c'est à dire un signal modulé en fréquence linéaire. Ce signal a une amplitude constante mais balaie tout le spectre de la bande passante de manière liénaire dans une période de temps définie. Cette technique de modulation sera détaillé plus loin dans le chapitre.

\vspace{0.1cm}

La technologie LoRa utilise également une technique de multiplexage en temps partagé ($TDMA$) pour permettre à plusieurs appareils de partager la même bande de fréquences de manière à maximiser l'utilisation de la capacité de transmission. Elle utilise également une technique de diffusion de données ($multicast$) pour envoyer les mêmes données à plusieurs appareils simultanément, ce qui permet de réaliser des économies de bande passante et d'énergie.

\vspace{0.1cm}

En plus de sa portée étendue et de sa faible consommation d'énergie, LoRa se distingue par sa sécurité de transmission, qui est assurée grâce à l'utilisation de codes de sécurité uniques et à la possibilité de chiffrer les données transmises. Elle est également compatible avec de nombreux protocoles de communication couramment utilisés dans l'IoT, tels que TCP/IP, HTTP et MQTT, ce qui facilite son intégration dans les systèmes existants.

\vspace{0.1cm}

Toutes ces  particularités font de LoRa une technologie complémentaire à celles déja existente plutot que rivale.
LoRa se compose de deux éléments principaux : la couche physique de la technologie et LoRaWAN, la couche MAC ($media$ $access$ $control$), une sous couche de la couche liaison de données dans le modèle $OSI$. la couche physique de LoRa gère la fréquence radio ainsi que la modulation. LoRaWAN gère les aspects réseaux comme la sécurité, la propagation, l'adressage et la sécurité).

\subsection{couche physique LoRa}

\subsubsection{Découpage de la couche physique}

\begin{figure}[h]
\centering

\includegraphics[scale=1]{images/physical_lora_rx.PNG}
\caption{Etapes de la transformation des données dans un éméteur LoRa}\label{term3}
\end{figure}


Les étapes de la conception de l'envoie de données dans la couche physique de LoRa sont les suivantes :

\vspace{0.1cm}

\begin{itemize}
\item Le codage de canal est une technique utilisée dans les systèmes de communication sans fil pour améliorer la robustesse et la fiabilité de la transmission des données. Dans le cas de LoRa, le codage de canal utilise la méthode $FEC$ ($Forward$ $error$ $correction$) pour corriger les erreurs causées par du bruit. La méthode FEC ajoute de l'information redondante sur les données.
\item Le mélange de canal (ou $channel$ $interleaving$) suit la codage de canal.  
Cette technique consiste à réarranger les bits de data avant de les transmettre, en les intercalant entre eux de manière à les disperser sur le spectre des fréquences de la transmission. Cela permet de réduire l'impact des erreurs consécutives ($burst$ $error$).
\item Le blanchiment de canal (ou $channel$ $whitening$) est la dernière étape avant la modulation du signal.
Cette technique consiste à utiliser une transformation aléatoire ou pseudo-aléatoire des données avant de les transmettre, de manière à répartir le spectre des fréquences de la transmission sur une large gamme de fréquences. Cela permet également la récupération d'horloge pour le récepteur.
\item La modulation CSS est l'étape principale de LoRa. En effet, les étapes précédantes sont communes à de nombreuses technologies, mais la particularité de LoRa provient de la modulation. Cette étape est détaillée dans la section suivante du chapitre.
\end{itemize}

\vspace{0.1cm}

Chacune des étapes décrites doit être inversément réalisée pour le récepteur. Ainsi pour la récupération de donnée à l'arrivée, l'appareil récepteur gère la démodulation, le déblanchiment, le démellement et de décodage.

\vspace{0.1cm}

Cette analyse a été faite en $reverse$ $engeneering$ (\textcolor{red}{citer article}). Le reverse engineering consiste à analyser un produit ou un système afin de comprendre comment il fonctionne ou d'identifier ses principes de conception. Dans le contexte de LoRa, le reverse engineering examine la technologie derrière LoRa afin de comprendre ses principes de base et sa conception.


\subsubsection{Modulation CSS}

Contrairement aux modulation classique en amplitude ou en fréquence, la modulation CSS étale le signal sur une large bande de fréquence. La modulation en fréquence est linéaire et utilise des chirps. un chirp est un signal dont la fréquence change en continue tout en conservant une amlitude constante. Il existe deux types de chirps : les $upchirp$ et $downchirp$.
Dans un upchirp la fréquence augmente avec le temps tandis que dans un downchirp la fréquence diminue.

\vspace{0.1cm}

le signal est donc séparé sur une large bande de fréquence, permettant par exmeple plusieurs transmission sans causer d'interférence.
La modulation CSS est l'une des principale contribution au fait que LoRa possède une faible consommation et une longue portée. Cette technioque est très bien intégrée aux appareils a faible puissance utilisé par les technologies LoRa.

\subsubsection{Spreading factor}

LoRa permet d'envoyer des paquets sur un longue distance à faible puissance. Selon l'environement dans lequel les appareils LoRa sont présents, il peut être utile de pouvoir ajuster certaines capacités.

\vspace{0.1cm}

L'étalement, ou $spreading$ $factor$ permet de déterminer le taux de variation de fréquence pour un signal. Modifier le spreading factor ajuste différentes propriétés de la communications. Par exemple, si on augmente le spreading factor, les quatre conséquences principales sont :

\vspace{0.1cm}

\begin{itemize}
\item l'augmentation de la portée. Comme la largeur de bande est plus large, la communication peut atteindre une portée supérieure.
\item Augmentation de la résistance aux interférences. Comme le signal est étalé sur une bande plus largeur, il y a moins de risque de subit des interférences.
\item Plus petit débit de données. Comme la bande est large, dans un temps défini moins de données sont transmises.
\item Plus faible consommation. Les données transmises à un taux plus faible consomment moins d'énergie, ce qui prolonge la durée de vie des appareils dont l'économie d'énergie est une priorité.
\end{itemize}

Diminuer le spreading factor engendre l'effet inverse.

\subsubsection{Structure d'un paquet LoRa}

un paquet LoRa est structuré en 5 parties différentes :

\vspace{0.1cm}

\begin{itemize}
\item Le préambule : la première partie du paquet, composée d'un nombre variable d'upchirps. La valeur par défaut est fixée à 8 upchirps.
\item L'identificateur réseau : après le préambule, le paquet contient deux symbole modulé pour l'indentification réseau.
\item Des symboles de synchronization de fréquence. Après l'identificateur réseau, il y a des downchirps pour faire la dinstinction entre les offsets d'échantillionage de temps ou de fréquence.
\item La tête ($header$) du paquet. Elle contient les informations relatives à la taille du paquet, le code rate, la présence ou non d'un CRC ($cyclique$ $redundancy$ $check$) et une checksum.
\item Le payload. La dernière partie du paquet contient le payload d'une taille maximale de 255 bits et l'éventuelle CRC de 16 bits.
\end{itemize}




\subsection{LoRaWAN}

LoRaWAN est un protocol de type $low$ $power$, $wide$ $area$ $network$ (LPWAN) désigné pour la communication longue portée. Ce protocole opère avec la technologie LoRa et lui fournit une infrastructure capable de maintenir une communication à longue portée et à faible cout dans l'$IoT$.

\subsubsection{aspets généraux de la technologie}

LoraWan bénéficie donc d'une faible puissance de consomation et d'une portée accrue. Elle est également efficace dans différents environement. Le signal est capable de pénétrer diverse terrains et structures.
Le déploiment d'un infrastructure LoRa ne nécessite pas de license, et son réseau peut être public ou privé. 

\vspace{0.1cm}

Le coeur de LoRaWAN réside sur la gestion de l'énergie, permettant aux appareils de fonctionner avec une consommation d'énergie minimale, prolongeant leur durée de vie tout en garantissant une fonctionnalité à long terme. A cette caractéristique de faible consommation d'énergie s'ajoute ses capacités en termes de portée, capable de pénéter diverses environements. Cela rend la technologie efficace aussi bien milieu rurale qu'urbain. LoRaWAN opère sur une bande de fréquence qui ne nécessite pas de license d'émition, par exemple sur la bande ISM pour \textit{Industrial, Scientific, and Medical}.Les bandes ISM, (868 MHz en Europe ou 915 MHz aux USA) sont disponible pour l'utilisation de différentae technologie, incluant LoraWAN.

\vspace{0.1cm}

LoRaWAN possède des capacités de géolocalisation, permettant au réseau de détecter et de localiser précisément les appareils au sein de son domaine. LoraWAn utilise différente méthodes pour localiser ses appareils comme \textbf{Received signal strengh indication} ou RSSI, \textit{Time difference on Arrival} ou TDOA, une triangulation ou alors une combinaison de plusieurs des méthodes.

\vspace{0.1cm}

LoRaWAN utilise des protocoles de sécurité end-to-end, aussi bien dans un réseau public intégré que dans un réseau privé.L'architecture LoraWan (décrite en détails dans la section 1.3.2.2) contient plusieurs couche de sécurité. Au niveau des \textit{end devices}, une routine d'identification est imposée avant l'accès au réseau. seul les appareils de confiance son donc autorisé à communiquer. Ensuite, une fois la communication commencée, les données sont chiffrées avant d'être transmise dans le réseau. Le framework sécuritaire de Lora ne se limite pas à l'autentification et au chiffrage. LoraWan gère également les mise à jour en continue par les airs, ainsi qu'une supervision continue sur d'éventuelles intrusions.

\vspace{0.1cm}

Avec toutes ces caractéristiques, LoraWan s'est développé dans de nombreux domaines aussi bien environementaux qu'industriels. Les principales utilisations de LoraWan actuelles sont les suivantes :

\vspace{0.1cm}

\begin{itemize}

\item la surveillance environementale en général. Lorawan peut être déployé pour surveiller des niveaux de températures, d'humidité, de bruits ou encore d'autres paramètres dans n'importe quel milieu. Une compagnie Hollandaise, Sensoterra, utilise notamment LoraWan pour surveiller la qualité des sols.
\item Les \textit{smart cities}. LoraWan est actif sur différents aspects comme la gestion intelligent de l'éclairage, la gestion des déchets, la surveillance, etc.
\item l'embarqué industriel. La maintenance et la surveillance de matériel et de l'équipement peut être gérée par Lorawan? TataSteel, une compagnie indienne, utilise LoraWan pour ces équipements industriels.
\item la prévention de catastrophe naturelle. Que ce soit en prévision, pendant ou après d'éventuelles catastrophes naturelles, la longue portée et la surveillance en temps réel sont des atoux cruciaux pour ce genre d'évènement.
\end{itemize}

\vspace{0.1cm}

Cependant, toutes ses caractéristiques entrainent un certain nombre de limitations. La restriction de la fréquence en fonction de la région peut rendre le déploiment d'une même infrastructure à différent endroit dans le monde plus difficile. Cela peut aussi entrainer des problèmes de compatibilité entre régions, notament pour des chaines logistique ou d'approvisionement qui traverse plusieurs régions.

\vspace{0.1cm}

Une faible consomation de puissance avec une grande portée a un impact sur la taille et la vitesse de l'information. La taille du payload d'un message est limitée entre 51 et 241 octets. La vitesse de transmission est également peu élevée, atteignant un maximum de 5.5kbps sur une largeur de bande de 125hz.

\vspace{0.1cm}

la communication au sein d'un réseau LoraWan se fait en grande partie de manière asynchrone. La synchronisation dépend de la classe de l'appareils, qui est détaillé dans la section topologie. C'est un avantage pour maintenir une grande autonomie de batterei pour les appareils. LoraWan possède un système pour limiter les colision entre message si plusieurs appareils communiquent simultanénent. Ce système est basé sur une combinaison entre \textit{Listen before talk} LBT et des delais aléatoires (citer papier). Il est néamoins possible que dans un environement très dense des collisions puisse encore se produire. La comunications asynchrone et le système d'évitement de collision entraine une augmentation du temps entre les envois et la réception de message.

\subsubsection{Topologie de Lorawan}

\begin{figure}[h]
\centering

\includegraphics[scale=0.1]{images/architecture.png}
\caption{Tolpologie de l'infrastructure LoraWan}\label{term4}
\end{figure}

La figure 4 montre les 4 types d'appareils qui composent la topologie d'un infrastrucure LoraWan. 

Les end devices sont les noeuds qui collectent les informations a envoyer à travers le réseau. Ils sont catégorisés en trois sous classes : A, B et C. Les apareils de classe A sont les plus économe en énergie. Ils ont été créé pour conserver leur énergie et comunique exclusivement en comunication asynchrone. Les appareils de classe A écoute les message provenant des serveur uniquement après avoir eux même transmis un message. la classe A regroupe les appareils les moins énergivore. Les appareils de classe B sont assez similaires avec ceux de classe A, mais sont occationellement synchronisés avec les servers du réseau. Ils possèdent des capacités supérieur de reception leur permettant de se synchroniser avec le schedule des serveurs, ce qui augmente considérablement l'efficacité du temps de réponses dans le réseau. Finalement, les appareils de classe C sont en écoute permanente de message provenant des serveur. Ils sont plus les plus réactif mais également les plus énergivores. Les end devices sont donc classés selon deux paramètres : leur réactivité et leur consomation d'énergie. En fonction de eur classe, ils ont également la possibilité de recevoir des message server après avoir transmis de l'information. L'envoie d'un message d'un end devices vers les serveurs est appelé \textit{uplink message} et l'envoie d'un message depuis les serveur vers les end devices est appelée \textit{downlink message}.

Les gateways font le role d'intermédiaire entre les end devices et le serveur réseau. Ils recoivent les transmissions depuis les end devices dans leur zone de couverture
et forward les messages vers le serveur réseau. Les gateways peuvent écouter plusieurs fréquence simultanénent (\textit{multichanneling}) là où les end devices n'écoutent qu'une seule fréquence. Les gateway gèrent la communication radio avec les end devices en utilisation la modulation de LoRa.

Le serveur réseau est la composante centrale de l'infrastructure. Il gère tout le réseau, que ce soit les données reçu des gateays, l'identification et l'activation des end devices dans le réseau, le routing ou encore l'adapation du data rate. Le serveur réseau supervise également l'aspect sécurité au sein du réseau en gèrant les clefs de chiffrage et les protocoles de sécurité.

Le serveur application de LoraWan reçoit les données forwardés depuis le serveur réseau. C'est l'interface entre le réseau de LoraWan et les différents service ou application d'utilisaturs finaux. Les utilisateurs intéragissent avec le serveur d'application pour n'importe quelle action a effectuer sur le réseau ou pour la récupération de données du réseau. Les données reçues par le serveur réseau sont traduite par le server d'application avant d'être interprétée par l'utilisateur final.

\subsubsection{Sécurité}

La sécurité dans l'architecture LoraWan se concentre sur trois axe principaux :

\begin{itemize}
\item l'authentification : qui communique avec qui.
\item L'intégrité : les données ne sont pas altérées entre l'émetteur et le recepteur.
\item La confidentialité : les données ne sont visible par personne au sein du réseau hormi l'émetteur et le récepteur. 
\end{itemize}

La sécurité repose sur le chiffrage des données. Les données sont chiffrés en utilisant l'algorithme de cryptographie AES (\textit{advanced encryption standart}).La taille des clefs est de 128 bits. Ce choix est motivés par un équilibre entre une sécurité suffisante et une consomation réduire de ressources.

Il y a deux types de clefs utilisée dans LoraWan. La \textit{root key} est la clé partagée entre un end device et le serveur réseau. Cette cle est utilisée pour l'authentification initiale et l'établissement d'une communication entre les deux éléments du réseau. Cette clé n'est jamais transmise par les air, elle est stockée dans un \textit{join server}. Un join server est un server dédié au contenu sensible à l'activation du matériel dans un réseau LoRaWAN. Il autentifie le réseau et les application du servers. Il gère les root keys. il génère également le second type de clés de LoraWan, les \textit{session keys}.

Les session keys sont des clé générée dynamiquement par le join server et utilisé durant l'échnage de donnée pendant une session. Il y a deux session key différente, la \textit{AppSKEY} pour le chiffrage des payload d'application, et la \textit{nwkSKEY} pour les fonctionalités du réseau (le chifrage à la couche MAC, les vérification d'integrité, etc).

\subsubsection{Session}

L'établissement d'une session entre un end devices et le réseau LoraWan peut se faire de deux façons différentes.

La première méthode est une méthode dynamique appelée \textit{Over the Air Activation} ou OTAA et se déroule de la façon suivante: 
\begin{itemize}
\item Le device possède initialement deux indentificateurs, un DevEUI et un appEUI.
\item La requête pour rejoindre le réseau est initiée par le end device. Il en envoir un message \textit{join request} au serveur réseau. La join request contient ses identificateurs, ainsi qu'un nombre aléatoire généré par le device.
\item Le serveur réseau accepte (ou décline) la requête et vérifie les identifiateurs du device dans ses enregistrements. 
\item le server génère ensuite un nombre aléatoire appelé \textit{DevNonce} et renvoir un message \textit{join accept} contenant le DevNonce, l'adresse du devices ainsi que les clefs (NwkSKey et appSKey) de session.
\item le en devices reçoit le message join accept. Il extrait les clefs envoyé et calcule ses propres clefs de session avec ses paramètres (les clefs envoyé par le join server ainsi que le devNonce).
\item le device fait maintenant parti du réseau. Chaque message que le device va envoyer au serveur sera chiffré avec ses clefs.
\end{itemize}
        
La seconde méthode est hardcodé et permet à un end devices de rejoindre directement le réseau sans passer par l'indentification. cette méthode est appelée \textit{activation by personalisation} ou ABP. Voici la procédure de la session :
\begin{itemize}
\item Le device possède à l'avance son adresse ainsi que ses clefs de session.
\item Le device est déploiyé au préalable dans la zone de couverture du réseau LoraWan.
\item Sans devoir itinialiser de procedure \textit{join request}, le end devices transmet directement ses données au serveur en utilisant ses clefs préconfigurées. L'échange de clef avec le serveur n'a pas lieu.
\end{itemize}

Cette seconde procèdure a comme avantage d'être plus rapide à exécuter car toute la partie d'initialisation est passée. Le processus d'initialisation peut être contraignant en ressource ce qui rend la méthode ABP moins énergivore. Cependant l'utilisation de clef hardcodée directement dans les devices est une pratique moins sécuritaire. Comme pour la taille de clefs, il y a un équilibre entre consommation d'énergie et sécurité.
