

\chapter*{Introduction}

\addcontentsline{toc}{chapter}{Introduction}
\renewcommand{\leftmark}{INTRODUCTION}

L'avènement de L'$Internet$ $of$ $Things$ a lancé une nouvelle ère d'appareils connectés, ouvrant de nouvelles possibilités de partage de l'information, d'automatisation et de protection. Bien que le concept lui même soit prometteur, la technologie qui l'accompagne est essentielle.
Les premières technoligies utilisées pour l'IoT étaient les technologies sans fil déja présentes comme le Wifi ou le Bluetooth. Performantes dans certains cas, elles étaient néanmoins limitées : une consommation en énergie élevée, une portée restreinte et parfois même un cout d'infractrucutre trop important.
Dans ces circonstance est apparu $LoRa$, une technologie développée en particulier pour l'IoT. Sa capacité à gérer les communications longue portée même dans des environements peu adaptés est une révolution pour le domaine.

\vspace{0.1cm}

L'expansion de L'IoT soulève une nouvelle problèmatique de sécurité. Entre autres, l'indentification des noeuds au sein des réseaux est essentielle. Il a été découvert que des noeuds fabriqués avec les mêmes microprocesseurs et modèles d'émetteurs-récepteurs radio peuvent présenter de subtiles particularités dans les caractéristiques de leurs signaux. Cette variabilité intrinsèque de la transmission des signaux radio peuvent être exploitées pour distinguer les noeuds d’un réseau. En écoutant leurs signaux radio émis et en analysant leurs signatures distinctes, il devient possible de les identifier.

\vspace{0.1cm}

Ce travail est structuré en quatre parties. Le premier chapitre sert d'aperçu global du signal radio afin d'y développer et rappeler les concept de télécommunication de base. Ce chaptire présente également les technologies LoRa et LoRaWAN à travers leurs caractéristiques et leur pertinence dans l'IoT.
Le second chapitre rassemble les travaux qui ont déja été effectués dans ce domaine.
Le troisième chapitre est dédié à l'étude expérimental du sujet. Les aspect pratique y seront appliqués, notamment l'utilisation de radio logicielle afin de capturer des signaux radio. Ces signaux seront ensuite analysé grace à diverse méthodes détaillé dans ce chapitre.
La dernière partie du travail présentera une présentation des résultats obtenu en suivant l'analyse effectuée au chapitre précédant. Enfin le travail sera achevé en concluant sur de potentielle implications plus larges à ce sujet ainsi que des recherches plus approfondies.