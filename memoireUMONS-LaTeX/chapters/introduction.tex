

\chapter*{Introduction}

\addcontentsline{toc}{chapter}{Introduction}
\renewcommand{\leftmark}{INTRODUCTION}

L'avènement de \textit{L'Internet of Things} (IoT) a lancé une nouvelle ère d'appareils connectés, ouvrant de nouvelles possibilités de partage de l'information, d'automatisation et de protection. Bien que le concept lui même soit prometteur, la technologie qui l'accompagne est essentielle.
Les premières technologies utilisées pour l'IoT étaient les technologies sans fil déja présentes comme le Wifi ou le Bluetooth. Elles ont cependant plusieurs limitations : une consommation en énergie élevée, une portée restreinte et parfois même un coût d'infrasctructure trop important.
Dans ces circonstances est apparu \textit{LoRa}, une technologie développée en particulier pour l'IoT. Sa capacité à gérer les communications longue portée même dans des environements urbains très denses est un grand atout pour le domaine.

\vspace{0.1cm}

L'expansion de L'IoT soulève une nouvelle problèmatique de sécurité. Entre autre, l'identification des noeuds au sein des réseaux est essentielle. Il a été découvert que des noeuds fabriqués avec les mêmes microprocesseurs et modèles d'émetteurs-récepteurs radio peuvent présenter de subtiles particularités dans les caractéristiques de leurs signaux. Cette variabilité intrinsèque de la transmission des signaux radio peut être exploitée pour distinguer les noeuds d’un réseau. En écoutant leurs signaux radio et en analysant leurs signatures distinctes, il devient possible de les identifier.

\vspace{0.1cm}

Ce travail est structuré en trois parties. Le premier          \hyperref[chap1]{chapitre} sert d'aperçu global du signal radio afin d'y développer et rappeler les concepts de base. Ce chapitre présente également les technologies LoRa et LoRaWAN à travers leurs caractéristiques et leur pertinence dans l'IoT.
Le deuxième chapitre est dédié à l'étude expérimental du sujet. Les aspects pratiques y seront appliqués, notamment l'utilisation de radio logicielle afin de capturer des signaux radio. Ces signaux seront ensuite analysés et automatisés pour la suite du travail.
Le troisième chapitre du travail présentera la méthode utilisée pour réaliser l'objectif du mémoire, ainsi que son application pratique sur les appareils décrits au chapitre précedent. Enfin le travail sera achevé en concluant sur de potentielles implications plus larges à ce sujet ainsi que des recherches plus approfondies.

\vspace{0.1cm}

Afin de bien comprendre les enjeux du mémoire, voici une petit historique de l'évoltion des préoccupations dans l'IoT des deux décénnies précedantes. Les technologie mais également le concept même d'\textit{Internet of Things} ont évolué depuis.

\vspace{0.1cm}

Bien que le concept d'appareils connectés remonte aux années 70, l'avènement de l'internet of thing arrive en fin de milénaire. Ce concept est associé à la technologie RFID \textit{Radio Frequency Identification}\cite{RFID}, qui permet d'utiliser les ondes radios afin d'identifier des objets ou des personnes. Le but initial était de rendre tout objet dans le monde identifiable par un code EPC ou \textit{Electronic Product Code}\cite{EPC}, un peu comme le code barre. Durant ces années, plusieurs entreprises lancent leurs premièrs appareils connectés. Tout cet enthousiame pour la connectivité met au second plan les questions de sécurité. Ainsi la première partie du développemnt de l'iot se concentre surtout sur la qualité de communication entre les objets plutot que sur leur sécurité.

\vspace{0.1cm}

Vers la fin des années 2000, l'augmentation du nombre d'appareils est si grande qu'elle a atteint tous les domaines de la société. Certains domaines étant plus critique que d'autre d'un point de vue sécurité (l'énergie, les transports, la santé, etc),l'intégrité des données, la confidentialité et les accès réseaux deviennent le centre de l'attetion. 
Le concept de certificats x.509, initalement développé pour le world wide web avant les années 2000, a un regain d'attention dans cette période. Plus largement, la structure de la technologie PKI (\textit{Public Key infrastructure}, qui utilise les certificats x.509)\cite{PKI} a été adaptée pour s'intégrer aux problématiques de l'embarqué. Un certificat est un document digital permettent de vérifier l'identité d'une entité, comme d'un appareil, un utilisateur ou une organisation. Il se base sur la liaison d'une clé public à l'entité établie par une \textit{Certificate Authority (CA)}. La CA agit en tant que tier de confiance et assure la légitimité de l'information grâce au certificat. Ainsi, les trois axes principaux de la sécurité dans l'IoT émergent : l'autentification, l'intégrité des données et la confidentialité.

\vspace{0.1cm}


Vers les années 2010, le nombre d'appareils connectés dépasse le nombre d'êtres humains, forçant une transition vers l'\textit{ipv6} tant le nombre d'appareils est évelé et continue d'augmenter. L'information a pris de la valeur et de l'ampleur. Ainsi, vient se greffer de nouveau enjeux économiques en plus des enjeux sécuritaires. La quantité de données générées nécessite de revoir le stockage de l'information. C'est ainsi que va apparaitre le Edge computing\cite{edge}, qui est une réponse directe aux besoin des architectures de gérer autan de données en périphérie de réseau. Le concept du edge computing vise à effectuer des calculs et des analyses des données directement sur les appareils connectés, plutôt que de les envoyer vers un centre de données centralisé. Cela réduit la latence, améliore l'efficacité du réseau et permet des analyses en temps réel. Le premier malware spécialemnt centré sur l'iot fait son apparition. Mirai\cite{Mirai} exploite une faille lui permettant de récupérer le mots de passe d'appareils afin de s'en servir pour lancer des attacks \textit{DDoS (distributed denial of service)} à grande échelle. En quelques années l'IoT est passé d'un gadget d'entreprise à un vériable enjeu économique et sécuritaire, centré autour de l'information. Les seules perspectives de législation concernant la sécurité de l'internet of thing n'apparaitront de tradivement en fin de décennies avec La loi européenne sur le \textit{Réglément générale sur la protection des données}(source : loi \href{https://commission.europa.eu/law/law-topic/data-protection/data-protection-eu_en}{RGPD}. cette loi ne couvre pas la sécurité des appareils mais plutot l'utilisation des données sur internet en général.

\vspace{0.1cm}

A partir de la fin des années 2010, le stockage et la transmission de données, l'authentification d'appareils ou encore la confidentialité sont au centre des préoccupations. Initialement utilisée dans les cryptomonnaies, la technologie Blockchain est un mécanisme de base de données qui permet un partage transparent des informations au sein d'un réseau. Une base de données Blockchain stocke les données dans des blocs qui sont reliés entre eux dans une chaîne. Les données sont chronologiquement cohérentes, car il n'est pas possible de supprimer ou modifier la chaîne sans le consensus du réseau. Par conséquent,la technologie Blockchain peur servir de livre inaltérable ou immuable pour le suivi des ordres, des paiements, des comptes et d'autres transactions. Le système dispose de mécanismes intégrés qui empêchent les entrées de transactions non autorisées et créent une cohérence dans la vue partagée de ces transactions. L'implémentation de la blockchain pour l'iot confère les avantages suivants\cite{block} : 
\begin{itemize}
\item l'immuabilité. la blockchain permet de créer un enregistrement immuable de toutes les interactions et communications des appareils. Cet enregistrement peut être utilisé pour détecter et empêcher l'accès non autorisé ou la modification des appareils ou des données dans l'IoT.
\item La décentralisation. il est possible de créer un système décentralisé pour l’authentification et la communication des appareils. Chaque appareil IoT se connecte au réseau blockchain et se voir attribuer une identité numérique unique, qui est vérifiée grâce à l'utilisation de signatures numériques ou de contrats intelligents. Cela élimine le besoin d’une autorité centrale pour authentifier les appareils et annule ainsi les risques de \textit{single point of failure}.
\item La confidentialité.La technologie Blockchain peut sécuriser la communication entre les appareils IoT grâce à l'utilisation de la cryptographie à clé publique ou asymétrique. Cela permet l’échange sécurisé d’informations entre appareils sans avoir recours à des intermédiaires.
\end{itemize}

\vspace{0.1cm}

Les menaces de sécurité sont de plus en plus sophistiquées au début des années 2020. Comment faire encore confiance aux infrastructures qui doivent gérer autant d'appareils ? La réponse est de ne plus leur faire confiance. Le zero Trust model est donc un modèle basé sur l'absence totale de confiance et une vérification constante, que la demande d'accès provienne de l'intérieur ou de l'extérieur du réseau. Dans un modèle de sécurité classique, une fois qu'un utilisateur ou un appareil accède au réseau interne, on lui fait souvent implicitement confiance, ce qui lui permet une grande liberté d'actions au sein du réseau. Le Zero Trust model suppose cependant que des menaces peuvent exister à la fois à l’intérieur et à l’extérieur du périmètre du réseau et nécessite donc une vérification continue de la confiance. Un modèle qui s'applique sur ce principe devrait contenir les éléments suivants\cite{zero1} :
\begin{itemize}
\item La vérification d'identité. Les utilisateurs et les appareils doivent subir une authentification avant d'accèder à n'importe quel service ou ressource du réseau.
\item Le \textit{Least privilege acces}. les permissions sont accordées de manière limités selon le besoin de l'utilisateur ou de l'appareil.
\item La micro segmentation. Diviser le réseau en segments pour limiter son accès par les appareils.
\item La surveillance en continue. L'analyse du traffic, du comportement et des activités des appareils.
\item le chiffrage des données.
\end{itemize}

\vspace{0.1cm}


Avec l'arrivée des ordinateurs quantiques dans les prochaines années, les méchanismes de chiffrage basés sur la complexité mathématique comme RSA ou ECC sont voués à disparaitre  \cite{quantumcrypto}. La puissance de calcul des ordinateurs quantiques est déja considérée comme une véritable menace pour la sécurité informatique. Fort heureusement, c'est également un nouveau champ de possibilité qui s'ouvre pour la sécurité, avec le développement du \textit{post quantum cryptography}. Un premier protocole résistant aux menances quantiques, \textit{Quantum Key Distribution} permet d'établir des canaux de communications entre différents appareils dans l'iot. Ce protol n'est pas encore en service dans l'iot, mais les premiers tests réalisés en laboratoire sont très prometteurs\cite{qinternet}.