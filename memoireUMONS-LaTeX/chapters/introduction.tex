

\chapter*{Introduction}

\addcontentsline{toc}{chapter}{Introduction}
\renewcommand{\leftmark}{INTRODUCTION}

L'avènement de \textit{L'Internet of Things} (IoT) a lancé une nouvelle ère d'appareils connectés, ouvrant de nouvelles possibilités de partage de l'information, d'automatisation et de protection. Bien que le concept lui même soit prometteur, la technologie qui l'accompagne est essentielle.
Les premières technologies utilisées pour l'IoT étaient les technologies sans fil déja présentes comme le Wifi ou le Bluetooth. Elles ont cependant plusieurs limitations : une consommation en énergie élevée, une portée restreinte et parfois même un coût d'infractrucutre trop important.
Dans ces circonstance est apparu \textit{LoRa}, une technologie développée en particulier pour l'IoT. Sa capacité à gérer les communications longue portée même dans des environements urbain très dense est un grand atout pour le domaine.

\vspace{0.1cm}

L'expansion de L'IoT soulève une nouvelle problèmatique de sécurité. Entre autres, l'identification des noeuds au sein des réseaux est essentielle. Il a été découvert que des noeuds fabriqués avec les mêmes microprocesseurs et modèles d'émetteurs-récepteurs radio peuvent présenter de subtiles particularités dans les caractéristiques de leurs signaux. Cette variabilité intrinsèque de la transmission des signaux radio peut être exploitées pour distinguer les noeuds d’un réseau. En écoutant leurs signaux radio et en analysant leurs signatures distinctes, il devient possible de les identifier.

\vspace{0.1cm}

Ce travail est structuré en quatre parties. Le premier          \hyperref[chap1]{chapitre} sert d'aperçu global du signal radio afin d'y développer et rappeler les concepts de base. Ce chapitre présente également les technologies LoRa et LoRaWAN à travers leurs caractéristiques et leur pertinence dans l'IoT.
Le second chapitre rassemble les travaux qui ont déja été effectués dans le domaine de l'indentification d'objets IoT.
Le troisième chapitre est dédié à l'étude expérimental du sujet. Les aspects pratiques y seront appliqués, notamment l'utilisation de radio logicielle afin de capturer des signaux radio. Ces signaux seront ensuite analysés grâce à diverses méthodes détaillées dans ce chapitre.
La dernière partie du travail présentera les résultats obtenus en suivant l'analyse effectuée au chapitre précédant. Enfin le travail sera achevé en concluant sur de potentielles implications plus larges à ce sujet ainsi que des recherches plus approfondies.